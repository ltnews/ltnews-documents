% !TEX root = ../../proyect.tex

\chapter{Planificación}\label{planificacion}
\section{Metodología}\label{sec:metodologia_scrum}

Todo proyecto que esté englobado dentro de la ingeniería del software posee una metodología que proporciona un marco de trabajo estructurado, planificado y controlado para el proceso de desarrollo. Sin embargo, no hay una única metodología a aplicar, sino más bien múltiples y variadas. Lo importante no es tanto escoger la \textit{perfecta}, que no existe, sino ver aquella o aquellas que mejor se ajusten a la naturaleza dle proyecto en cuestión. 

Como se ha podido observar anteriormente, nos hace falta una metodología que agilice el desarrollo dada la tipología del proyecto. Es por tanto imprescindible el uso de una metodología \textit{agile} o ágil. Esta se orienta a la construcción y entrega. Para conseguir esto, reduce el tiempo de desarrollo. Dado que el producto no se puede reducir conceptualmente, habrá que construir por fases para conseguir esto.

Dicho lo anterior, se puede apreciar que los procesos de especificación, diseño y la misma implementación son concurrentes. Utiliza, por ello, un ciclo de vida evolutivo, en el que el ciclo requisitos-desarrollo-evaluación termina en una versión del proyecto. Esta puede ser probada por el cliente, dando mayor oportunidad a feedback sobre el entendimiento de los requisitos.

La metodología ha sido elegida por la la naturaleza del proyecto, como se ha dicho. Además, al sector al que va dirigido es muy concreto y, a la vez, diferente. Así, no se conocían a priori todos los requisitos, porque no se tenían unos clientes concretos a los que preguntar que abarcasen la representación del conjunto. Esto se ha podido ver con propiedad cuando se ha hablado de las entrevistas con \textit{early adopters}, en el punto \ref{sec:iteraciones}.

Así, los requisitos, se han ido conociendo a lo largo del desarrollo del trabajo. Se ha intentado que esto fuera mínimo siguiendo la metodología Running Lean, aunque los métodos preventivos nunca previenen al completo sobre posibles problemas. Así, como se decía, sin todos los requerimientos por parte del usuario, no es posible comenzar un ciclo de vida clásico, ya que en este es necesario conocer, idealmente, todos los requisitos del proyecto.

Por otra parte, se necesitaba tener, en poco tiempo, una versión testeable de la aplicación, para obtener un flujo de información más claro sobre lo que se desea realmente del aplicativo. Este es el segundo argumento a favor de la metodología ágil, ya que trabaja continuamente con versiones de la aplicación, al acabar el ciclo antes mencionado, y estas pueden ser fácilmente enseñables a futuros clientes para recibir opiniones.

Sim embargo, como se ha dicho, \textit{ninguna metodología es perfecta}. \textit{Agile} presenta algunos aspectos negativos, como son la difícil planificación por los requisitos cambiantes o la falta de cobertura en todas las áreas del proceso, entre otras. No obstante, se asumen por la naturaleza del proyecto, en favor de los motivos mencionados más arriba.

\section{Planificación temporal}\label{sec:planificacion_temporal}

\section{Roles del proyecto}\label{sec:roles_proyecto}

\section{Planificación económica}\label{sec:planificacion_economica}