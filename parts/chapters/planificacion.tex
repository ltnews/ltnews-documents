% !TEX root = ../../proyect.tex

\chapter{Planificación}\label{planificacion}
\section{Metodología}\label{sec:metodologia_scrum}

Todo proyecto que esté englobado dentro de la ingeniería del software posee una metodología que proporciona un marco de trabajo estructurado, planificado y controlado para el proceso de desarrollo. Sin embargo, no hay una única metodología a aplicar, sino más bien múltiples y variadas. Lo importante no es tanto escoger la \textit{perfecta}, que no existe, sino ver aquella o aquellas que mejor se ajusten a la naturaleza dle proyecto en cuestión. 

Como se ha podido observar anteriormente, nos hace falta una metodología que agilice el desarrollo dada la tipología del proyecto. Es por tanto imprescindible el uso de una metodología \textit{agile} o ágil. Esta se orienta a la construcción y entrega. Para conseguir esto, reduce el tiempo de desarrollo. Dado que el producto no se puede reducir conceptualmente, habrá que construir por fases para conseguir esto.

Dicho lo anterior, se puede apreciar que los procesos de especificación, diseño y la misma implementación son concurrentes. Utiliza, por ello, un ciclo de vida evolutivo, en el que el ciclo requisitos-desarrollo-evaluación termina en una versión del proyecto. Esta puede ser probada por el cliente, dando mayor oportunidad a feedback sobre el entendimiento de los requisitos.

La metodología ha sido elegida por la la naturaleza del proyecto, como se ha dicho. Además, al sector al que va dirigido es muy concreto y, a la vez, diferente. Así, no se conocían a priori todos los requisitos, porque no se tenían unos clientes concretos a los que preguntar que abarcasen la representación del conjunto. Esto se ha podido ver con propiedad cuando se ha hablado de las entrevistas con \textit{early adopters}, en el punto \ref{sec:iteraciones}.

Así, los requisitos, se han ido conociendo a lo largo del desarrollo del trabajo. Se ha intentado que esto fuera mínimo siguiendo la metodología Running Lean, aunque los métodos preventivos nunca previenen al completo sobre posibles problemas. Así, como se decía, sin todos los requerimientos por parte del usuario, no es posible comenzar un ciclo de vida clásico, ya que en este es necesario conocer, idealmente, todos los requisitos del proyecto.

Por otra parte, se necesitaba tener, en poco tiempo, una versión testeable de la aplicación, para obtener un flujo de información más claro sobre lo que se desea realmente del aplicativo. Este es el segundo argumento a favor de la metodología ágil, ya que trabaja continuamente con versiones de la aplicación, al acabar el ciclo antes mencionado, y estas pueden ser fácilmente enseñables a futuros clientes para recibir opiniones.

Sim embargo, como se ha dicho, \textit{ninguna metodología es perfecta}. \textit{Agile} presenta algunos aspectos negativos, como son la difícil planificación por los requisitos cambiantes o la falta de cobertura en todas las áreas del proceso, entre otras. No obstante, se asumen por la naturaleza del proyecto, en favor de los motivos mencionados más arriba.

Aunque se ha hablado continuamente de la metodología ágil y su aplicación al proyecto, esto no es tan directo. Es necesario descender a marcos de trabajo que especificen y complementen dichas formas de trabajo. Las elegidas para este proyecto serán SCRUM y Kanban, ambas complementarias, que se explicarán en los siguientes puntos.

\subsection{SCRUM}
Dentro de la metodología anteriormente comentada, es necesario descender a los detalles del día a día. De esta manera surge SCRUM como un marco de trabajo que permite el trabajo conjunto entre miembros de un mismo equipo. Su etimología proviene del rugby, y al igual que él, a la hora de entrenar, se ha de basarse en la experiencia, organizarse para resolver los problemas y reflexionar sobre los resultados obtenidos para mejorar.

SCRUM es un marco que provee a cualquier equipo de trabajo de unas pautas para poder aplicarse. Este describe un conjunto de funciones, herramientas y reuniones que hacen trabajar de manera coordinada a los miembros del equipo. Dicho lo anterior, se puede ver que SCRUM es un marco de trabajo, mientras que la metodología ágil es un ideal. Por ello no se puede trabajar directamente con \textit{agile}, sino que es necesario recurrir a operativas que implementan, o al menos, desarrollan dicha mentalidad.

Este marco de trabajo se basa en el aprendizaje y adaptación al medio. Esta heurística hace al equipo humilde, ya que reconoce que no posee el pleno conocimiento al inicio, y que irá evolucionando a lo largo de la realización del proyecto. Hace, de manera soterrada, que el equipo de trabajo se vaya adaptando a los cambios tanto de las condiciones como del cliente. Esto se consigue presentando a los interesados continuas versiones del producto o servicio que se va desarrollando, además de un orden por prioridad de las peticiones del cliente.

\figura{0.5}{img/scrum/scrum_learn}{Aprendizaje en SCRUM}{fig:scrum_learn}{}

Como se puede apreciar en la imagen, el aprendizaje de SCRUM sigue la estructura: IDEA-CONSTRUCCIÓN-DESPLIEGUE-APRENDIZAJE. Es foco lo tiene el aprendizaje, y no tanto el desarrollo. Sin embargo, no es este un marco rígido. SCRUM se adapta sin problema a cualquier tipo de proyecto, estableciendo únicamente algunas pautas a seguir. Lo que de verdad importa, y a esto va dirigido este marco, es hacer que la comunicación sea clara, la transparencia sea real y la dedicación sea continua.


\subsection{Kanban}

\section{Planificación temporal}\label{sec:planificacion_temporal}

\section{Roles del proyecto}\label{sec:roles_proyecto}

\section{Planificación económica}\label{sec:planificacion_economica}