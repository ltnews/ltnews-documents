% !TEX root = ../../proyect.tex

\chapter{Planificación}\label{planificacion}
\section{Metodología}\label{sec:metodologia_scrum}

Todo proyecto que esté englobado dentro de la ingeniería del software posee una metodología que proporciona un marco de trabajo estructurado, planificado y controlado para el proceso de desarrollo. Sin embargo, no hay una única metodología a aplicar, sino más bien múltiples y variadas. Lo importante no es tanto escoger la \textit{perfecta}, que no existe, sino ver aquella o aquellas que mejor se ajusten a la naturaleza dle proyecto en cuestión. 

Como se ha podido observar anteriormente, nos hace falta una metodología que agilice el desarrollo dada la tipología del proyecto. Es por tanto imprescindible el uso de una metodología \textit{agile} o ágil. Esta se orienta a la construcción y entrega. Para conseguir esto, reduce el tiempo de desarrollo. Dado que el producto no se puede reducir conceptualmente, habrá que construir por fases para conseguir esto.

Dicho lo anterior, se puede apreciar que los procesos de especificación, diseño y la misma implementación son concurrentes. Utiliza, por ello, un ciclo de vida evolutivo, en el que el ciclo requisitos-desarrollo-evaluación termina en una versión del proyecto. Esta puede ser probada por el cliente, dando mayor oportunidad a feedback sobre el entendimiento de los requisitos.

La metodología ha sido elegida por la la naturaleza del proyecto, como se ha dicho. Además, al sector al que va dirigido es muy concreto y, a la vez, diferente. Así, no se conocían a priori todos los requisitos, porque no se tenían unos clientes concretos a los que preguntar que abarcasen la representación del conjunto. Esto se ha podido ver con propiedad cuando se ha hablado de las entrevistas con \textit{early adopters}, en el punto \ref{sec:iteraciones}.

Así, los requisitos, se han ido conociendo a lo largo del desarrollo del trabajo. Se ha intentado que esto fuera mínimo siguiendo la metodología Running Lean, aunque los métodos preventivos nunca previenen al completo sobre posibles problemas. Así, como se decía, sin todos los requerimientos por parte del usuario, no es posible comenzar un ciclo de vida clásico, ya que en este es necesario conocer, idealmente, todos los requisitos del proyecto.

Por otra parte, se necesitaba tener, en poco tiempo, una versión testeable de la aplicación, para obtener un flujo de información más claro sobre lo que se desea realmente del aplicativo. Este es el segundo argumento a favor de la metodología ágil, ya que trabaja continuamente con versiones de la aplicación, al acabar el ciclo antes mencionado, y estas pueden ser fácilmente enseñables a futuros clientes para recibir opiniones.

Sim embargo, como se ha dicho, \textit{ninguna metodología es perfecta}. \textit{Agile} presenta algunos aspectos negativos, como son la difícil planificación por los requisitos cambiantes o la falta de cobertura en todas las áreas del proceso, entre otras. No obstante, se asumen por la naturaleza del proyecto, en favor de los motivos mencionados más arriba.

Aunque se ha hablado continuamente de la metodología ágil y su aplicación al proyecto, esto no es tan directo. Es necesario descender a marcos de trabajo que especificen y complementen dichas formas de trabajo. Las elegidas para este proyecto serán SCRUM y Kanban, ambas complementarias, que se explicarán en los siguientes puntos.

\subsection{SCRUM}
Dentro de la metodología anteriormente comentada, es necesario descender a los detalles del día a día. De esta manera surge SCRUM como un marco de trabajo que permite el trabajo conjunto entre miembros de un mismo equipo. Su etimología proviene del rugby, y al igual que él, a la hora de entrenar, se ha de basarse en la experiencia, organizarse para resolver los problemas y reflexionar sobre los resultados obtenidos para mejorar.

SCRUM es un marco que provee a cualquier equipo de trabajo de unas pautas para poder aplicarse. Este describe un conjunto de funciones, herramientas y reuniones que hacen trabajar de manera coordinada a los miembros del equipo. Dicho lo anterior, se puede ver que SCRUM es un marco de trabajo, mientras que la metodología ágil es un ideal. Por ello no se puede trabajar directamente con \textit{agile}, sino que es necesario recurrir a operativas que implementan, o al menos, desarrollan dicha mentalidad.

Este marco de trabajo se basa en el aprendizaje y adaptación al medio. Esta heurística hace al equipo humilde, ya que reconoce que no posee el pleno conocimiento al inicio, y que irá evolucionando a lo largo de la realización del proyecto. Hace, de manera soterrada, que el equipo de trabajo se vaya adaptando a los cambios tanto de las condiciones como del cliente. Esto se consigue presentando a los interesados continuas versiones del producto o servicio que se va desarrollando, además de un orden por prioridad de las peticiones del cliente.

\figura{0.5}{img/scrum/scrum_learn}{Aprendizaje en SCRUM}{fig:scrum_learn}{}

Como se puede apreciar en la imagen, el aprendizaje de SCRUM sigue la estructura: IDEA-CONSTRUCCIÓN-DESPLIEGUE-APRENDIZAJE. Es foco lo tiene el aprendizaje, y no tanto el desarrollo. Sin embargo, no es este un marco rígido. SCRUM se adapta sin problema a cualquier tipo de proyecto, estableciendo únicamente algunas pautas a seguir. Lo que de verdad importa, y a esto va dirigido este marco, es hacer que la comunicación sea clara, la transparencia sea real y la dedicación sea continua.

\subsubsection{Artefactos de SCRUM}
Los artefactos son aquellos desarrollos que realizamos para solucionar el problema. En SCRUM son tres: \textit{Product Backlog}, \textit{Sprint Backlog} e \textit{Increment}.

El \underline{\textit{Product Backlog}} es la lista de tareas que debe realizar el equipo para desarrollar con éxito el entregable al cliente. Esta es una lista con fluctuaciones en la que se encuentran requisitos, mejoras y correcciones a realizar sin indicar a priori una planificación de la misma. Cada tarea posee una prioridad que irá cambiando dependiendo de múltiples factores: cliente, entorno, sistema, etc.

El \underline{\textit{Sprint Backlog}} es una lista de elementos a realizar en un periodo concreto propuesto por el equipo. Esta se alimenta del \textit{Product Backlog} y se encuentra cerrada durante el sprint. De hecho, antes de iniciar este periodo de tiempo, en la reunión de planificación del sprint, el equipo elige aquellas tareas del producto en las que trabajará.

El \underline{\textit{Increment}} es el producto desarrollado al final del sprint. Este está estrechamente relacionado con el concepto de finalizado. En este sólo se incluirán los elementos del \textit{Sprint Backlog} que se hayan finalizado durante el sprint. Será el equipo de desarrollo el que indique qué entiende él mismo sobre la finalización de una tarea. Este prototipo construido, lejos del producto final, podrá ser testeable por el cliente y así proponer cambios.

Dentro de los artefactos, se suelen incluir también los gráficos de análisis de SCRUM, como \textit{Burn-down Chart} o \textit{Burn-up Chart}. Estos se podrán realizar tanto sobre el \textit{Product Backlog} como por el \textit{Sprint Backlog}. Estos son indicadores de cómo se ajusta la planificación a la realidad en base a las tareas finalizadas.

\subsubsection{Eventos de SCRUM}

Los eventos son el conjunto de protocolos, reuniones y acontecimientos secuenciales que los equipos de SCRUM realizan de manera periódica. Se encuentran seis eventos principales, que se detallarán a continuación.

\figura{}{img/scrum/scrum_planning}{Eventos en SCRUM}{fig:scrum_planning}{}

El primero es la \underline{Organización del \textit{Backlog}}. Su función es dirigir el producto final hacia una visión comercial y tecnológica que busque el éxito del proyecto. El mantenimiento del \textit{Product Backlog} se realiza con los comentarios de los usuarios y del equipo de desarrollo para priorizar los elementos de la misma. Gracias a esta tarea, cada elemento de la lista estará a punto para poder trabajar en cualquier momento.

El \underline{\textit{Sprint Planning}} es la reunión en la que el equipo de desarrollo planifica el trabajo que va a realizar durante el Sprint. En esta se establece un objetivo y se añaden al \textit{Sprint Backlog} los elementos pertinentes. Al final de esta reunión, cada miembro debe de tener claro cuál va ser el incremento al final del sprint y la manera de lograrlo.

El \underline{Sprint} es el periodo de tiempo real en el que equipo SCRUM trabaja para llevar a término el incremento. El intervalo de tiempo será elegido por el equipo, aunque autores recomiendan entre una semana y un mes, dependiendo de la tipología del proyecto. Este será el punto de referencia para todo el marco de SCRUM.

El \underline{\textit{Daily Meeting}} es la reunión diaria en la que todos los miembros del equipo indican el estado de sus tareas e inquietudes que posean sobre la consecución de las mismas. La duración ha de ser breve, en torno a un cuarto de hora.

El \underline{\text{Sprint Review}} es la reunión al final del sprint. En esta, el equipo comprueba el incremento realizado y lo muestra a las partes interesadas con el objetivo que obtener feedback sobre las tareas que ha ido realizando. La duración variará dependiendo del número de semanas del sprint. Se recomiendan entre una y cuatro horas.

Por último, el \underline{\textit{Sprint Retrospective}}. Es la reunión donde el equipo indica aquellos aspectos que han funcionado o no del sprint. Este se enfoca en todos los aspectos que rodean al equipo, desde el desarrollo hasta las personas. Su duración variará dependiendo de la duración del sprint, pero suele estar entre una y tres horas.

\subsubsection{Funciones en SCRUM}
El equipo de SCRUM debe poseer entre sus miembros de tres funciones principales: \textit{Product Owner}, \textit{Scrum Master} y Equipo de desarrollo. Además, como los equipos SCRUM suelen ser multidisciplinares, estos se podrán estructurar a su vez en otros equipos. Los tres roles SCRUM se concretarán a continuación.

El \underline{\textit{Product Owner}} es la persona que con mayor profundidad conoce el producto. Su misión es entender los requisitos del mercado, de los clientes y de la empresa. Estos se encargan de gestionar el \textit{Product Backlog}, deciden si se lanzan los incrementos y lideran las mayoría de las reuniones. Este no ha de ser el Jefe de Proyecto, ya que poseen misiones diferentes.

El \underline{\textit{SCRUM Master}} es la persona con mayor conocimiento de SCRUM en el equipo. Su misión es proporcionar formación al equipo, buscar formas de afinar en su práctica y asegurar su buen funcionamiento. Este posee un nivel de detalle amplio sobre el trabajo cualquier miembro del equipo. Por ello, puede ayudar al equipo a mejorar su transparencia y flujo de entrega.

Por último, se encuentra el \underline{Equipo de desarrollo}. Son los que realizan el trabajo de construcción en el día a día. Poseen el conocimiento de las prácticas orientada a un desarrollo sostenible. Estos han de tener una relación cercana, se deben encontrar en el mismo lugar y han de ser pocos. Autores expertos recomiendan entre cinco y siete.

Cada uno de los miembros del equipo, posee diferentes cualidades. Es tarea de todos conocerlas y optimizarlas, evitando cuellos de botellas. Es por ello importante sus funciones de autorganización y trabajo en equipo. Gracias a este doble aspecto, estiman la duración de sus tareas en base al histórico de tareas realizadas.

\figura{0.75}{img/scrum/scrum_team}{Equipo en SCRUM}{fig:scrum_team}{}

\subsection{Kanban}
Kanban, al igual que SCRUM, es un marco de trabajo dentro de la metodología \textit{agile}. Este marco pone su foco en un tablero donde aparecen las tareas que está realizando el equipo y el estado de las mismas. Este da, de un simple vistazo, gran parte de la información del estado del proyecto. Además, debido a su simpleza, puede ser aplicable a cualquier tipo de industria.

Gracias a este doble aspecto de simpleza y diversidad, es enormemente usado dentro del marco de SCRUM. Simplemente añade algunas consideraciones al uso del tablero. En concreto, se especifican cuatro acciones y seis prácticas, todas orientadas a realizar las tareas pendientes de manera eficaz.

\subsubsection{Principios en Kanban}
David Anderson es considerado como el líder del pensamiento Lean y Kanban. Este formuló los cuatro principios para llevar a cabo de manera correcta hacia el proceso evolutivo e incremental.

El primer principio: empezar con lo que se hace ahora. Como se ha dicho, Kanban no requiere ningún tipo de configuración, pudiendo ser aplicado sobre cualquier flujo de trabajo. Así, para empezar a usarlo, no es necesario realizar cambios radicales.

El segundo principio: comprometerse a buscar e implementar cambios incrementales y evolutivos. Este marco de trabajo está diseñado para enfrentarse a cualquier resistencia al cambio. Todo está orientado a cambios continuos evolutivos e incrementales del proceso actual. Se evitan cambios drásticos en aras hacia pequeños virajes en el rumbo.

El tercer principio: respetar los procesos, las responsabilidades y los cargos actuales. Kanban reconoce que los anteriores roles y procedimientos pueden tener valor en el entorno del proyecto, por eso quiere conservarlos. Como se ve, se incita al cambio, aunque no se prescribe de manera alocada.

El cuarto principio: animar el liderazgo en todos los niveles. Recuerda a cada miembro del equipo que la consecución de los fines son es exclusiva a las capas directivas. Es por ello que promueve pequeños actos día a día de mejora continua en todos los aspectos del equipo. Este se pondrá en valor cuando se consiga un rendimiento óptimo.

\figura{}{img/scrum/scrum_kanban}{Ejemplo de tablero Kanban}{fig:scrum_kanban}{}

\subsubsection{Prácticas en Kanban}
Una vez visto los puntos claves de la mentalidad hacia Kanban, se han de realizar las tareas pertinentes para implementar con éxito el marco de trabajo. En concreto, David Anderson indica seis prácticas para conseguir dicho fin.

La primera práctica: visualizar el flujo de trabajo. Se han de definir dos aspectos importantes. El primero es el estado por el que pasan las tareas, desde que se definen hasta que finalizan. El segundo es la información que poseerá cada tarea en el tablero.

La segunda práctica: Eliminar las interrupciones. Uno de los perjuicios más importantes que se intentan solventar con Kanban es el continuo cambio de enfoque en los miembros del equipo. Así, en cada columna del tablero, se indicará un número máximo de tareas que pueden estar a la vez. Este es el \textit{Work In Progress} (WIP). Con esto conseguimos que solo se puede empezar con una tarea cuando alguna se termine.

La tercera práctica: gestionar el flujo. La idea de seguir la Kanban no es otra que crear un proceso continuo e ininterrumpido de las tareas. Cada uno de los estadios del flujo son las columnas por las que pasan las tareas. Teniendo esto claro, habrá que analizar la velocidad y continuidad del movimiento.

La cuarta práctica: hacer las políticas explícitas. Hay una máxima en el mundo Lean: no puede mejorar algo que no se entiende. Así, todo proceso dentro del marco deberá estar definido, publicado y promovido.

La quinta práctica: circuitos de retroalimentación. Si se quiere que el cambio tenga éxito y sea duradero, se han de tener en cuenta las reuniones regulares para la transferencia de conocimiento. Ejemplos de estas son las reuniones diarias para ver el estado de las tareas o la reunión de revisión de entregas.

La sexta práctica: mejorar colaborando. Finalmente, si se quiere que este cambio exceda a la jurisdicción propia del proyecto que se está ejecutando y percuta en la empresa, es necesario compartir ideas entre los miembros del equipo. En concreto, se ha de buscar una visión conjunta de las teorías sobre la manera de trabajar, el flujo de las tareas o procesos internos.

\subsection{Aplicación}
Una vez analizadas las metodología de trabajo, se indicarán cómo se han llevado a la práctica durante la ejecución y estudio del Trabajo Fin de Máster. Se ha utilizado un derivado entre SCRUM y Kanban, haciendo especial hincapié en el primero.

Parte importante de SCRUM es el equipo, gracias al cual se optimizan flexibilidad, creatividad y productividad. Normalmente, se trabajan con los tres roles anteriormente mencionados. Dado que solo hay un miembro que lleva a cabo todo el proceso, \textit{Product Owner}, \textit{Scrum Master} y \textit{Development Team} se unen en una persona.

De la misma manera que en la definición del equipo, ha habido modificaciones en los eventos de SCRUM. La primera es el Sprint, que en nuestro caso son dos semanas. Estas son dos semanas en tiempo acumulado, no real. Esto se debe a que la dedicación al TFM ha sido fluctuante a lo largo del tiempo, además centradas en fines de semana.

Las reuniones, como se puede ya deducir, son de poco tiempo, aunque con pautas bien establecidas, como se puede ver a continuación.
\begin{itemize}
    \item \textit{Daily Scrum}: antes de ponerse a trabajar, se establece en el Tablero Kanban las tareas en estados \textit{Ready} e \textit{In Progress}.
    \item \textit{Sprint Review}: se tiene con el profesor tutor del proyecto de manera formal y con un tiempo establecido (sobre una hora), y con los potenciales clientes, de manera informal y sin tiempo previamente establecido.
    \item \textit{Sprint Retrospective}: se sacan las gráficas para ver resultados, se apuntan las cosas a mejorar y se toman decisiones concretas, como, por ejemplo, mejorar una tarea, cambiar una funcionalidad fruto del feedback recibido.
    \item \textit{Sprint Planning Meeting}: se hacen inmediatamente después de la anterior reunión estableciendo en el Tablero los requisitos a cumplir, así como lo que haya que mejorar del anterior Sprint.
\end{itemize}

Como se ha dicho anteriormente, el punto neurálgico de SCRUM en este proyecto es el Tablero Kanban. Este es la referencia visual del estado de las tareas y, por ende, del proyecto. Es usado tanto por todos los interesados en el proyecto y, por ello, se basa en su buen uso el éxito del proyecto.

El Tablero se usa para saber en qué se está trabajando y qué falta por hacer. En este se establecen las tareas a hacer, así como un tiempo estimado. El tiempo estimado se mide en una escala tomada según la secuencia de Fibonacci, donde 1 representa una tarea muy sencilla de una media hora de duración y 20 (nuestro máximo), una tarea excesivamente larga, de unas diez horas. Se evitará siempre tener tareas de más de 8 puntos, aunque, en algunos casos, no es viable.

En el Tablero usado se poseen los siguientes flujos, en cuanto a las columnas se refiere.
\begin{itemize}
    \item \textit{Backlog}: al principio del Sprint, aquí aparecen todas las tareas a realizar, así como aquellas que se dejen para otros Sprints.
    \item \textit{Ready}: aparecen las tareas que hay que completar en dicha semana.
    \item \textit{In progress}: son las tareas que se están realizando. Como limitación, el tiempo estimado de todas las tareas que aparecen en esta columna, no podrá superar los 8 puntos, si hay más de una tarea.
    \item \textit{Done}: son las tareas hechas, esto es, desarrolladas y probadas.
\end{itemize}

\figura{}{img/scrum/scrum_github}{Tablero Kanban usado}{fig:scrum_github}{}

Se ha usado para ello la plataforma GitHub, que ofrece a desarrolladores un tablero kanban con las características previamente mencionadas. Las tareas serán las \textit{issues} de todos los repositorios anejos al mismo. En concreto, las tareas de los repositorios que vuelcan a este tablero son los códigos de Back-End, Front-End y documentación asociada.


\section{Planificación temporal}\label{sec:planificacion_temporal}

En este apartado se encuentra tanto la planificación como la duración real, una vez finalizado el desarrollo. Como se ha dicho, la planificación utilizada ha sido utilizando la metodología ágil. A continuación, se recoge la planificación por sprints.

\cuadro{lr}{Planificación de Sprints}{tab:sprints}
{
    \textbf{Nombre} & \textbf{Estimación} \\

    S0 - Fase previa & 70:00:00 \\
    S1 - Investigacion & 30:00:00 \\
    S2 - Funcionalidad completa & 100:00:00 \\
    S3 - Virtualización de los servicios & 30:00:00 \\
    S4 - Capa Inteligencia Artificial & 70:00:00 \\
}

A su vez, los sprints se han de desglosar en fases concretas en cuanto al desarrollo del producto se refiere. Por ello, la planificación más coherente con respecto a cualquier producto software es la siguiente.

\cuadro{llr}{Planificación temporal}{tab:planificacion}
{
    \textbf{Fase} & \textbf{Subfase} & \textbf{Estimación} \\

    \textbf{1. Planificación} & & \textbf{10:00:00} \\
    \textbf{2. Estudio} & & \textbf{30:00:00} \\
     & 2.1. Estudio de LEAN & 10:00:00 \\
     & 2.2. Realización de modelos & 5:00:00 \\
     & 2.3. Entrevistas & 10:00:00 \\
     & 2.4. Estudio de los datos & 5:00:00 \\
     \textbf{3. Análisis} & & \textbf{50:00:00} \\
     & 3.1. Documentación & 40:00:00 \\
     & 3.2. Elicitación de requisitos & 5:00:00 \\
     & 3.3. Modelo conceptual & 5:00:00 \\
     \textbf{4. Diseño} & & \textbf{20:00:00} \\
     & 4.1. Detalles técnicos & 10:00:00 \\
     & 4.2. Modelo de datos & 5:00:00 \\
     & 4.3. Mockups & 5:00:00 \\
     \textbf{5. Formación} & & \textbf{30:00:00} \\
     & 5.1. Django Rest Framework & 5:00:00 \\
     & 5.2. VueJS & 10:00:00 \\
     & 5.3. ElasticSearch & 10:00:00 \\
     & 5.4. Docker & 5:00:00 \\
     \textbf{6. Implementación} & & \textbf{100:00:00} \\
     & 6.1. Back-End & 25:00:00 \\
     & 6.2. Front-End & 50:00:00 \\
     & 6.3. AI-Layer & 25:00:00 \\
     \textbf{7. Pruebas} & & \textbf{25:00:00} \\
     \textbf{8. Despliegue} & & \textbf{25:00:00} \\
     \textbf{9. Presentación} & & \textbf{10:00:00} \\
     \textbf{TOTAL} & & \textbf{300:00:00} \\
}

\section{Roles del proyecto}\label{sec:roles_proyecto}

Tal y como dijimos antes, en nuestro proyecto se unifican los roles de SCRUM en una única persona. Sin embargo, esto no quita que esta persona tenga que asumir a lo largo del proyecto una serie de roles, en función del tipo de tareas a realizar. Sin embargo, para simplificar dichos roles, se han unificado en la responsabilidad de Jefe de Proyecto.

Por ello, las responsabilidad adquiridas y sus misiones serán las que aparecen a continuación.
\begin{itemize}
    \item Jefe de proyecto: encargado de la gestión y planificación del proyecto, de reunirse con el cliente y de tomar decisiones sobre el equipo.
    \item Analista: encargado de recolectar los requisitos y necesidades del proyecto, así como la de diseñar el sistema de información.
    \item Desarrollador: se encarga de implementar los requisitos, es decir, de desarrollar el software.
    \item Tester: su misión es la asegurar la calidad de los productos obtenidos, en concreto, se encarga de elaborar y ejecutar las pruebas.
\end{itemize}

\section{Planificación económica}\label{sec:planificacion_economica}

Para la planificación económica se han considerado los gastos asociados a la ejecución del proyecto, sin ingresos asociados. Aunque el proyecto buscará en un futuro tener beneficios no entra este aspecto dentro del ámbito del Trabajo Fin de Máster.

Habrá que considerar, por tanto, dos tipos de costes, dependiendo si hacen referencia a las personas físicas, o a materiales e infraestructuras necesarias para llevar a cabo el proyecto.

\subsection{Costes directos}
Dentro de este tipo de costes se consideran exclusivamente los costes de personal. Tal y como se ha dicho en el apartado \ref{sec:roles_proyecto}, para este es necesario una serie de roles, cada uno de los cuales posee un salario distinto.

Así, dentro de estos costes se van a tener en cuenta los sueldos de los distintos tipos de trabajadores durante el desarrollo del proyecto. Para esto, nos se ha de basarse en el Boletín Oficial del Estado publicado en abril de 2009, referente a salarios mínimos, \cite{boe_salarios}. En nuestro caso, utilizaremos los datos del sector de las TIC.

Para todos los casos, se considerará que un año pose<e 250 días hábiles, que supone 2.000 horas. Ahora, analicemos cada rol en concreto.

\begin{itemize}
    \item Jefe de Proyecto: su sueldo mínimo es de 36.600\$ que a la hora hace 18,3\$
    \item Analista: su sueldo mínimo es de 21.555,66\$ que a la hora hace 10,78\$
    \item Desarrollador: su sueldo mínimo es de 15.442,56\$ que a la hora hace 7,72\$
    \item Tester: su sueldo mínimo es de 11.773,02\$ que a la hora hace 5,89\$
\end{itemize}

Además, a estos costes, hemos de aplicar impuestos que paga la empresa para cualquier trabajador en España. Todos son porcentajes que se aplican a la base de cotización. Estos son los siguiente:
\begin{itemize}
    \item Contingencias comunes
    \begin{itemize}
        \item Estos dan cobertura a bajas temporales o prestaciones de jubilación, entre otras.
        \item Supone el 23,6\%
    \end{itemize}
    \item Fondo de Garantía Social
    \begin{itemize}
        \item Es un organismo encargado de dar garantía a los trabajadores de su salario o indemnizaciones.
        \item Supone el 0,2\%
    \end{itemize}
    \item Formación Profesional
    \begin{itemize}
        \item Asegura la formación al trabajador.
        \item Supone el 0,7\%
    \end{itemize}
    \item Seguridad Social
    \begin{itemize}
        \item Dependerá del tipo de contrato y de su duración. Consideramos duración determinada y a tiempo parcial.
        \item Supone 7,7\%
    \end{itemize}
\end{itemize}

Teniendo en cuenta todos los porcentajes, nos queda para aplicar a la base de cotización una suma de los anteriores porcentajes: 32,2\%.