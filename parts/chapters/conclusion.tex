% !TEX root = ../../proyect.tex

\chapter{Conclusión}\label{conclusion}

\section{Resultados}\label{sec:resultados}

Como se ha ido pudiendo comprobar a lo largo de la memoria, la metodología Lean, en concreto, Running Lean, ha surtido efecto a la hora de enfocar el desarrollo de la aplicación. Creo que esto es debido a la antropología usada por dicha metodología. El centro del desarrollo ya no es el producto sino el cliente. Esto ha permitido testear desde el inicio la idea \textit{sobre papel}, sin necesidad de construir nada. Esto conlleva la reducción de costes y mayor implicación por parte del equipo de desarrollo en \textit{ponerse en los zapatos} del cliente.

Sin embargo, esto no es una acción puntual de un momento, como puede ser el estudio o análisis de la idea, debe ser un proceso que vaya desde el inicio de la concreción de la idea hasta la búsqueda del feedback continuo ante cualquier cambio en la aplicación. Es por tanto un actitud a asimilar, no una serie de pasos a realizar. Creo que esto es lo más complicado de entender a la hora de aplicar la metodología. En mi caso personal, creo que lo he aprehendido gracias a la lectura detenida del libro.

Un resultado negativo a tener en cuenta es la valoración de los comentarios de los \textit{early adopter}. En el caso de LT-News, me ha sido difícil diferenciar algunas características esenciales del Mínimo Producto Viable de las imprescindibles en base a aportaciones de los futuros usuarios. En aras de contentar a todos, creo que he caído en manos un servilismo por parte de dicho sector. Afortunadamente, esta actitud ha sido descubierta por mis tutores y he sabido podar correctamente en la fase de Estudio.

Resumidamente, creo que es un gran aporte la utilización de Running Lean para el estudio de las ideas de aplicación. Aporta conocimiento y enfoque diferentes al equipo de desarrollo. Sin embargo, creo que es imprescindible poseer a una persona que tutele el proceso para poder realizarlo correctamente y con objetividad.

\section{Mejoras futuras}\label{sec:mejoras_futuras}

\section{Lecciones aprendidas}\label{sec:lecciones_aprendidas}