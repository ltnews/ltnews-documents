% !TEX root = ../../proyect.tex

\chapter{Conclusión}\label{conclusion}

\section{Resultados}\label{sec:resultados}

Como se ha podido comprobar a lo largo de la memoria, la metodología LEAN, en concreto, \textit{Running Lean}, ha surtido efecto a la hora de enfocar el desarrollo de la aplicación. Creo que esto es debido a la antropología usada por dicha metodología. El centro del desarrollo ya no es el producto sino el cliente. Esto ha permitido testear desde el inicio la idea \textit{sobre papel}, sin necesidad de construir nada. Esto conlleva la reducción de costes y mayor implicación por parte del equipo de desarrollo: \textit{ponerse en los zapatos} del cliente.

Sin embargo, esto no es una acción puntual de un momento, como puede ser el estudio o análisis de la idea. Debe ser un proceso que vaya desde el inicio de la concreción de la idea hasta la búsqueda del feedback continuo ante cualquier cambio en la aplicación. Es por tanto un actitud a asimilar, no una serie de pasos a realizar. Creo que esto es lo más complicado de entender a la hora de aplicar la metodología. En mi caso personal, creo que lo he aprehendido gracias a la lectura detenida del libro de Ash Maurya.

Un resultado negativo a tener en cuenta es la valoración de los comentarios de los \textit{early adopter}. En el caso de LT-News, me ha sido difícil diferenciar algunas características esenciales del Mínimo Producto Viable de las prescindibles, en base a aportaciones de los futuros usuarios. En aras de contentar a todos, creo que he caído en manos un servilismo por parte de dicho sector. Afortunadamente, esta actitud ha sido descubierta por mis tutores y he sabido podar correctamente en la fase de Estudio.

Resumidamente, creo que es un gran aporte la utilización de \textit{Running Lean} para el estudio de las ideas de aplicación. Aporta conocimiento y enfoque diferentes al equipo de desarrollo. Sin embargo, creo que es imprescindible poseer a una persona que tutele el proceso para poder realizarlo correctamente y con objetividad.

\section{Mejoras futuras}\label{sec:mejoras_futuras}

La aplicación construida es un Mínimo Producto Viable, que, de momento, no busca una monetización y unos rendimientos económicos. Analizando el proceso seguido y hacia donde puede llegar la aplicación distingo tres fases para realizar en un futurible. Antes de ir desgranando una a una, se hará un informe de la situación actual del proyecto, para poder encarrilar correctamente dichos cambios.

El sistema se ha intentado realizar siguiendo las pautas y herramientas que propone la comunidad del software, utilizando últimas versiones de dichas herramientas y una arquitectura novedosa y profesional. Por tanto, se han cimentado las bases sobre futuros desarrollos.

A corto plazo, en menos de un año, la aplicación habrá adquirido un pequeño volumen de usuarios y una cantidad, no tan pequeña, de noticias. En este punto, pienso que sería conveniente realizar los siguientes cambios:

\begin{itemize}
    \item Empezar a integrar la información de las noticias con las redes sociales. Dado que este podría ser un requisito que se convierta en un mar sin orillas, se propone realizar tres acciones.
    \begin{itemize}
        \item Integrar las cuentas de Twitter de los periódicos añadidos y relacionar noticias con tweets para analizar y usar el movimiento de comentarios que genera, así como añadir noticias de actualidad.
        \item Añadir en cada noticia los botones de las principales redes sociales para poder distribuir fácilmente las noticias y hacer auto-publicidad de la plataforma.
        \item Permitir el registro con las cuentas de las redes sociales, así como su integración posterior. De esta manera, publicar contenidos automáticamente en estas como respuesta a una noticia, por ejemplo.
    \end{itemize}
    \item Otro aspecto importante debe ser la adición de la publicidad en la plataforma. Esta se deberá añadir intentando que esté integrada con el contenido de la aplicación sin ser molesta por su exceso a los usuarios. Además, se relacionará con Google Analytics para mostrar qué contenidos proveen mayor beneficios.
    \item Mostrar en la noticias información de contexto de la misma. Dado que esto puede ser muy amplio, se podría empezar integrando Wikipedia y mostrando en la misma noticia aquellos términos interesantes con la información que provea la enciclopedia.
    \item En este momento, será casi semanal el cambio que sufrirá el software. Para ahorrar tiempo, se proponer la realización de una plataforma de DevOps que facilite el desarrollo y recorte el \textit{time to market}.
\end{itemize}

A medio plazo, en torno a tres años, y si la aplicación va siendo popular, aumentará el número de usuarios considerablemente. Por ello, en este momento, se proponen los siguientes cambios:

\begin{itemize}
    \item En primer lugar, añadir un nuevo actor del sistema: analista. Este será, normalmente, un periodista de un medio o un redactor de un blog con un volumen medio de gente. Poseerá un acceso a un cuadro de mando de temas que le podrán hacer diferencial así como poder estudiar a sus competidores. El dashboard se caracterizará por las siguientes funciones.
    \begin{itemize}
        \item Este se basará en las noticias que se vayan leyendo y el perfil de aquellos que interaccionen con ella.
        \item Estarán las noticias más leídas del tema en cuestión así como los medios de los que proceden estas.
        \item Además, habrá una evolución en el tiempo tanto de la publicación de estas como de su popularidad por parte de los lectores.
        \item Poseerá un sobrecoste añadido debido a su gran aportación.
    \end{itemize}
    \item El flujo de feedback continuo ya poseerá un gran número de actores. Es por ello que ya no valdrán canales como el correo para la resolución de problemas o propuesta de nuevas características. En este momento, habrá que añadir un gestor de incidencias anexo a la aplicación, que permita un proceso centralizado, ordenado y visible.
    \item Al crecer tanto el número de usuarios, será necesario pensar en hacer la web más accesible a ellos. Es por ello que en dicho momento se planteará la posible realización de una aplicación híbrida. Esta dará enormes ventajas, entre ellas la sencillez de uso y una utilización mejor de las notificaciones.
    \item Dado que en este momento se estará popularizando la aplicación, será necesario ir marcando la diferencia con respecto a la competencia. Es por ello que se empezarán a estudiar técnicas avanzadas de NPL, como pueden ser el autoresumen y la traducción.
    \item Dado que van a ser cada vez mayores las características del portal sobre la competencia, se propone empezar en este momento un plan premium. Así, se podrá diferenciar las funcionalidades y dar aquellas mejores a los usuarios que paguem. Aun esto, es importante cuidar a los clientes que vienen gratuitamente a nuestra aplicación.
\end{itemize}

Por último, en un plazo aun mayor, de entre cinco a diez años, la aplicación será una de las referentes en el mercado, con una comunidad de usuarios en tres continentes y acogiendo a millones de usuarios. Será en este momento cuando se potenciarán tres características.

\begin{itemize}
    \item En primer lugar, desarrollar un dashboard por periódico o medio que esté en la aplicación. Albergará los mismo datos que el cuadro de mando por temas con una característica más: temáticas más recurrentes. Además, se centrará en mayor medida en los tipos de lector. Este se ofrecerá a analistas de la aplicación a un coste mayor, dada su embergadura. Estará pensado en editores jefes de periódicos físicos interesados en aumentar su audiencia.
    \item En este punto, y como guinda al pastel del plan freemium, se mejorará el dashboard personal de cada usuario. Con esta medida, se hará ver a los lectores que son importantes y se potenciará el uso de la IA.
    \item Por último, se ofrecerá una API con algunos datos de la aplicación que se vean convenientes. Esta poseerá un coste que se estudiará en el momento. La complejidad fundamental de esto serán las implicaciones legales.
\end{itemize}

\section{Lecciones aprendidas}\label{sec:lecciones_aprendidas}

Al margen de los resultados y conclusiones obtenidas a la aplicación de la metodología LEAN me gustaría concluir con algunos hechos que me han llevado a aprender con la puesta en práctica del presente Trabajo.

En primer lugar, indicar que la realización de este TFM me ha hecho madurar como ingeniero. Esto se debe a que el centro del trabajo no residía en el producto, sino en la metodología que se ha llevado a cabo para realizar el mismo. Como he ido diciendo en los puntos que venían al caso: me he dado cuenta de que importa más el cliente que el producto, no solo en teoría, sino en la praxis.

En un segundo punto, me gustaría hablar del reto que ha supuesto para mí, desde un punto de vista más tecnológico, la realización de la aplicación. Quería realizar una sistema que pudiese usarse en un entorno productivo real y creo que lo he conseguido. Para llegar a esto, he tenido que empaparme de muchas herramientas y tecnologías novedosas, donde había más ganas que referencias en la red. Esto me ha hecho madurar mucho como informático así como aprender a trabajar con gran parte de incertidumbre tecnológica.

Por último, creo que este tipo de proyectos necesita de un pequeño equipo para llevarse a cabo. Trabajar individualmente es bonito y agiliza ciertas tareas, pero en general, es arduo. No tanto por la acumulación de trabajo, sino por la falta de personas y lo que estas aportan: moral, sentimientos o puntos de vista. Veo cada vez más imprescindible en el mundo del software el calor humano que ofrece un equipo.