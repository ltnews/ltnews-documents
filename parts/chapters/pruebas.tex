% !TEX root = ../../proyect.tex

\chapter{Pruebas}\label{pruebas}
\section{Pruebas unitarias}\label{sec:pruebas_unitarias}

Los test unitarios prueban una funcionalidad en toda su completitud y tipología de casuísticas. Debido a la complejidad y gran tiempo que conlleva la realización de las pruebas, se han recudido estas a una parte representativa y en la parte del servidor.

Para desarrollarlas, por tanto, se ha utilizado el framework de pruebas que proporciona Django. Con él, solamente haciendo uso de la función \textit{Test}, se pueden ejecutar todas las pruebas a la vez. Se han comprobado principalmente los usos principales sobre una parte de la aplicación.

En total se han realizado en torno a 20 prueba unitarias de dos entidades exclusivamente: \textit{Section} y \textit{Profile}. Estas poseen toda la lógica que se ofrece por la API probadas, haciendo uso de una nueva base de datos para ello. Además, contienen tanto las pruebas positivas y negativas. Eto quiere decir que de estas entidades de comprueban el correcto funcionamiento, así como los posibles que podrían acontecer.

\section{Pruebas integradas}\label{sec:pruebas_integradas}

Como se ha dicho, los tests unitarios comprueban una funcionalidad concreta del sistema. Aunque hubiera pruebas sobre cada característica concreta del sistema, esto no implicaría la comprobación total de la aplicación. Es por ello importante la comprobación de todas las funcionalidades en su conjunto. Estos son las pruebas integradas.

\figura{0.5}{img/pruebas/travis}{Muestra de logs de Travis}{fig:travis}{}

Estas se realizan en la aplicación utilizando un sistema externo: Travis. Este, con un fichero de configuración, visto en el apartado \ref{sec:estructra_proyecto}, compone la arquitectura del sistema y realiza todo el conjunto de pruebas de manera global. Así, se consigue integrar toda la funcionalidad en una misma ejecución.

La siguiente imagen muestra los diferentes logs de las últimas ejecuciones de Travis. Se puede ver que hay algunos en error. Esto es debido a el cambio de composición de la aplicación: pasa de el montaje de cada componente individualmente a una servicios interconcectados y virtualizados con Docker.

\section{Pruebas de aceptación}\label{sec:pruebas_aceptacion}

Se ha hablado anteriormente de la figura del \textit{early adopter} como una pieza fundamental dentro del estudio y comprobación de la idea. Estas personas selectas, debido a su rol, se han encargado también de ir testando paultatinamente la aplicación. No de manera fría y accidentada, buscando fallos o faltas de usabilidad. Más bien han intentado usar la aplicación desde un primer momento en aras a convertirla en su futura referencia en cuanto a noticias se refiere.

Dicha utilización ha sido regular, aunque diferente en cada uno de ellos. En algunos consitía en un uso esporádico cada varios días. En otros se ha convertido casi en una costumbre semanal o mensual. Esto ha servido de acicate para la mejora de la calidad de los resultados y el acabado desde el punto de vista del diseño.

Estos, además, daban su opinión de cambios acontecidos en la aplicación. Gracias a esto, se ha podido ir modificando las diferentes características del sistema y se ha tenido en un periodo de tiempo corto el feedback de estos perfiles. Como botón de muestra de su aportación encontramos: el diseño de los colores, la latencia de determinadas vistas o la sugerencia del resumen de noticias.

Esto ha proporcionado al sistema de un flujo de comunicación directa con los clientes. En un futuro se estandarizará dicho proceso y se ampliará a cualquier usuario. Se hará como herramienta de feedback imprescindible para un desarrollo basado en la metodología Running Lean, como el indicaba en el punto \ref{sec:metodologia_lean}. Así, será el flujo completo: desde el lanzamiento de una nueva característica hasta el estudio de la recepción de la misma para controlar la gestión del cambio.

\section{Estándares de código}\label{sec:estandares_codigo}

Uno de los aspectos más importantes a la hora de llevar a cabo un proyecto dentro del área de la ingeniería del software es la calidad. De hecho, gran parte del éxito de una aplicación consiste en su calidad, en mayor o menor medida. No solo es importante cumplir los requisitos, con pruebas que nos aseguren su correcto funcionamiento. Es necesario también cumplir alguna serie de estándares o llevar el estilo del código a un determinado modelo. Esto nos asegura el seguimiento de ir en el buen camino dentro de un framework o lenguaje concreto. Esto, además, nos llevará a tener en cuenta cuestiones de seguridad, rendimiento o tamaño.

Este ha sido uno de los aspectos enfocados en la calidad que más se ha tenido en cuenta a la hora de realizar el proyecto. Para ello, se han utilizado dos herramientas principales: Optimize que provee los IDE de Jetbrains y Codacy. Del primero decir que, al igual que cualquier IDE, en PyCharm se nos muestran errores de código en tiempo de compilación, posibles errores en tiempo de ejecución, así como advertencias a la hora de usar una función obsoleta. Sin embargo, con esta herramienta, podremos ver qué recomienda Python para nombrar las variables, cómo realizar de manera correcta un try/except, así como muchísimas más características.

Otra herramienta utilizada es Codacy \footnote{Es posible acceder a los cuadros de mando de calidad indicando el repositorio en el enlace https://app.codacy.com/project/andjimrio/<repo>}. Esta analiza cualquier repositorio Git por cada commit que detecte, analizando tanto errores sintácticos, como problemas de seguridad que poseamos. Así, se ha intentado mejorar los repositorios del proyecto hasta llegar a la máxima certificación.

\figura{}{img/pruebas/codacy_repositories}{Respositorios en Codacy}{fig:codacy_repositories}{}

Como se ha dicho, Codacy comprueba varios aspectos. Uno de ellos es la seguridad. En este nos dice si nuestro sistema posee precauciones sobre determinado tipo de ataques. Como se puede observar, todo está correcto, debido a la potencia de los frameworks utilizados.

\figura{0.75}{img/pruebas/codacy_security}{Seguridad en Codacy}{fig:codacy_security}{}