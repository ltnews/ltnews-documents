% !TEX root = ../../proyect.tex

\chapter{Diseño}\label{diseno}
\section{Patrones}\label{sec:patrones}

En este apartado se mencionarán todos los patrones que se han utilizado para llevar a cabo este proyecto. Estos son soluciones concretas a problemas reales, abstrayendo tecnología y herramientas. Están enfocadas a la arquitectura del desarrollo, por lo que es imprescindible que se posean desde el principio.

Como se ha dicho, son soluciones a problemas ya estudiados y analizados desde el comienzo del software. Por ello, su utilización dará un valor añadido al código que lo hará más mantenible y reutilizable.

A continuación se encuentran los patrones utilizados. Están descritos de manera global, tal y como se puede encontrar en cualquier documentación. Esto quiere decir que no se descenderá a detalles de implementación.

\subsection{Modelo-Vista-Controlador}
El Modelo-Vista-Controlador, MVC en adelante, es un patrón de diseño que estructura el código en tres componentes principales: datos, lógica de negocio e interfaz de usuario. Para ello, propone la construcción del software diferenciando bien estos tres elementos.

\begin{itemize}
    \item \textbf{Modelo}: su misión es la representación de la información con la que se opera que incluirá tanto los datos como la lógica para operar con ellos.
    \item \textbf{Vista}: se encarga de presentar los datos de forma adecuada para interactuar con los mismo desde la interfaz de usuario.
    \item \textbf{Controlador}: tiene como fin el hacer de intermediario entre los dos otros componentes antes mencionados.
\end{itemize}

Este patrón es enormemente eficaz para la reutilización del código entre distintos módulos del sistema y la separación del software por su misión. Esto maximiza la cohesión y reduce el acoplamiento. Además, hace que el código sea más comprensible y facilita la labor de mantenimiento. Es por estos motivos por los que se ha escogido dicho patrón.

\subsection{Controlador-Servicio}
El Controlador, como se ha dicho, es el punto neurálgico de la aplicación. Este implementará la lógica de la aplicación. Por ello, es este el que accede al modelo de datos. Con estos, transformados y adaptados, construimos la plantilla de datos, que será transferida a la Vista.

Dada la importancia de este hecho, será necesario mantener ordenado el controlador. Si no es así, se puede convertir en un \textit{totum revolutum}: en un mismo método se puede encontrar una consulta a base de datos, una manipulación de dicho datos y su entrega posterior a una plantilla.

Una manera de simplificar el controlador es delegar alguna de sus funciones en otros componentes. Esto no es más que crear nuevos elementos que tengan una misión y se encarguen exclusivamente de la misión para la que han sido creados. Esta es una práctica cada vez más común dentro del marco del desarrollo global. Botón de muestra es la creación de arquitecturas basadas en microservicios: componentes creados para una misión \textit{ad hoc}.

Así, el servicio tendrá una doble misión. La primera es la interacción con la base de datos. Se convertirá así para el controlador en una interfaz programática o API. Será transparente para el Controlador todo lo concerniente a la gestión de los datos: conexión y desconexión con la base de datos, consultas y actualizaciones de datos, concurrencias e hilos. No es que el servicio se encargue de todo esto, sino que es él el que interactúa con el cliente de base de datos correspondiente para ofrecer dichas funcionalidades.

La segunda misión del servicio es ofrecer unos datos acordes para trabajar con ellos. Para ello, se suelen transformar los resultados en objetos. A esta técnica se la llama: \textit{Data Transfer Object} o DTO. Esto facilita la interacción con los datos, ya que son tratados como objetos del propio lenguaje.

En aras a lo cual, el Controlador usa al Servicio para toda la gestión con la base de datos. Con esta delegación, solo es necesaria la manipulación de los datos y la entrega a la Vista. Además, el construir una API interna aporta al código mayor versatilidad y cohesión, ya que se encontrarán agrupados los ficheros concernientes a la relación con la base de datos.

\subsection{Desacoplamiento Cliente-Servidor}
Ha sido un tema bastante estudiado la interacción entre lenguajes de programación de servidor en el cliente. Desde el inicio de las aplicaciones web, este problema se solucionó integrando herramientas al servidor que permitiesen añadir en tiempo de ejecución código al cliente. De esta filosofía, por ejemplo, nace PHP, lenguaje que se integra completamente dentro del HTML. Al igual ocurre con Java y sus JPA o con Python y sus Jinja Template.

Sin embargo, a medida que evoluciona el frontend, este se va aportado valor a sí mismo. Se empiezan a utilizar herramientas de servidor, pero para la construcción del mismo. Dicho de otro modo, es ahora en la compilación donde se consigue un código de cliente coherente y funcional. Es este el motivo, por el que se empiezan a construir aplicaciones de cliente de manera hermética a cualquier servidor.

La herramienta de servidor por excelencia, como se ha llamado, es Node. Esta consigue construir aplicaciones de cliente independientes trabajando exclusivamente con las tecnologías de cliente: HTML, CSS y JavaScript. Sin embargo, a estas se les da un valor añadido al poseer este doble aspecto: tecnologías de servidor en tiempo de compilación y tecnologías de cliente en tiempo de ejecución. Por ello, todo el código que se va construyendo se transforma y empaqueta (que no compila realmente) para que sea ejecutado por el navegador.

Como se ve, este hecho, cada vez más recurrente por otra parte, hace más clara esta división entre cliente y servidor. En primer motivo, dada su naturaleza; en segundo, dada su finalidad. Aunque desde el principio la naturaleza de ambos fuera diferentes, había puntos de unión que podían confundir. Ahora, que están totalmente diferenciados, es necesario respetar su esencia.

Con vistas al respeto a la naturaleza, se observa un cambio diametral: su finalidad. Mientras el servidor ofrece una API, es el cliente quien la consume. Como se puede observar, ahora puede haber más de un cliente asociado a un servidor. Además, todos poseerán el mismo punto de entrada para acceder a los datos y lógica de la aplicación. Esto hará más mantenible el código, así como aportará más frescura al mismo.

\section{Modelo de datos}\label{sec:modelo_datos}

El modelo de datos utilizado para la base de datos se ha basado en el definido en el punto anterior: \ref{sec:modelo_conceptual} Modelo conceptual. Este diagrama ha sido generado en base a las relaciones que se poseen. Dado que anteriormente se axplican las entidades y sus relaciones, ahora será conveniente detenerse más en las propiedades y detalles técnicos.

\figura{0.75}{img/diagram/model_diagram}{Modelo de datos}{fig:model_diagram}{}

A continuación, se irá mencionando cada una de las clases del modelo así como sus atributos.

\begin{itemize}
    \item \underline{User}: es, como dijimos anteriormente, la entidad que gestionará a todos los usuarios. Algunos de sus atributos más relevantes son los siguientes.
    \begin{itemize}
        \item Atributos
        \begin{itemize}
            \item \underline{username}: nombre de usuario único en el sistema.
            \item \underline{email}: correo electrónico.
            \item \underline{password}: hash de la contraseña.
            \item \underline{first\_name}: nombre.
            \item \underline{last\_name}: apellidos.
            \item \underline{last\_login}: fecha de última conexión.
            \item \underline{is\_superuser}: booleano que es verdadero si el usuario es administrador.
            \item \underline{is\_active}: booleano que es cierto cuando el usuario esté activo.
            \item \underline{date\_joined}: fecha el que se registró.
        \end{itemize}
        \item Relaciones
        \begin{itemize}
            \item \underline{profile}: relación \underline{OneToOne} con \underline{Profile}.
        \end{itemize}
    \end{itemize}
    \item \underline{Profile}: en esta entidad se guarda toda la información relevante del perfil del usuario.
    \begin{itemize}
        \item Atributos
        \begin{itemize}
            \item \underline{image}: link de la imagen del usuario.
        \end{itemize}
        \item Relaciones
        \begin{itemize}
            \item \underline{sections}: relación \underline{OneToMany} con \underline{Section}.
            \item \underline{comments}: relación \underline{OneToMany} con \underline{Comment}.
            \item \underline{statuses}: relación \underline{OneToMany} con \underline{Status}.
            \item \underline{keywords}: relación \underline{ManyToMany} con \underline{Keyword}.
        \end{itemize}
    \end{itemize}
    \item \underline{Section}: representa las secciones del usuario. Esta es una agrupación de feeds RSS. Por ello, cada sección tendrá asociado un usuario y uno o varios periódicos; estos, además, podrán estar en otras secciones.
    \begin{itemize}
        \item Atributos
        \begin{itemize}
            \item \underline{title}: título.
            \item \underline{description}: descripción.
        \end{itemize}
        \item Relaciones
        \begin{itemize}
            \item \underline{user}: relación \underline{ForeignKey} con \underline{Profile}.
            \item \underline{feeds}: relación \underline{ManyToMany} con \underline{Feed}.
        \end{itemize}
    \end{itemize}
    \item \underline{Feed}: esta entidad representa a todos los feeds o periódicos online. Esta poseerá varias noticias y, como dijimos anteriormente, podrá estar en varias secciones.
    \begin{itemize}
        \item Atributos
        \begin{itemize}
            \item \underline{title}: nombre del feed.
            \item \underline{link\_rss}: enlace al feed RSS.
            \item \underline{link\_web}: enlace al link de la web del feed.
            \item \underline{description}: descripción del feed.
            \item \underline{language}: idioma del feed.
            \item \underline{logo}: link del logo.
        \end{itemize}
        \item Relaciones
        \begin{itemize}
            \item \underline{sections}: relación \underline{ManyToMany} con \underline{Section}.
            \item \underline{items}: relación \underline{OneToMany} con \underline{Item}.
        \end{itemize}
    \end{itemize}
    \item \underline{Item}: representa a cada noticia. Esta, como ya hemos visto, posee numerosas relaciones, dado que es el núcleo de nuestro sistema.
    \begin{itemize}
        \item Atributos
        \begin{itemize}
            \item \underline{title}: título de la noticia.
            \item \underline{link}: enlace.
            \item \underline{description}: pequeña descripción que provee el feed RSS.
            \item \underline{image}: imagen de la noticia.
            \item \underline{article}: cuerpo de la noticia.
            \item \underline{pubDate}: fecha de publicación.
            \item \underline{creator}: autor.
        \end{itemize}
        \item Relaciones
        \begin{itemize}
            \item \underline{feed}: relación \underline{ForeignKey} con \underline{Feed}.
            \item \underline{comments}: relación \underline{OneToMany} con \underline{Comment}.
            \item \underline{statuses}: relación \underline{OneToMany} con \underline{Status}.
            \item \underline{keywords}: relación \underline{ManyToMany} con \underline{Keyword}.
        \end{itemize}
    \end{itemize}
    \item \underline{Status}: entidad que representa que estado posee una noticia para un usuario concreto. Todos sus atributos son booleanos; por defecto, se inicializarán a falso.
    \begin{itemize}
        \item Atributos
        \begin{itemize}
            \item \underline{view}: el usuario ha visto la noticia.
            \item \underline{read}: el usuario ha leído la noticia o, por lo menos, ha entrado en ella.
            \item \underline{web}: el usuario ha accedido a la web original de la noticia.
            \item \underline{like}: el usuario ha marcado esta noticia como ‘Me gusta’.
            \item \underline{saves}: el usuario ha guardado la noticia.
        \end{itemize}
        \item Relaciones
        \begin{itemize}
            \item \underline{user}: relación \underline{ForeignKey} con \underline{Profile}.
            \item \underline{item}: relación \underline{ForeignKey} con \underline{Item}.
        \end{itemize}
    \end{itemize}
    \item \underline{Comment}: representa un comentario de un usuario a una noticia.
    \begin{itemize}
        \item Atributos
        \begin{itemize}
            \item \underline{description}: cuerpo del comentario.
            \item \underline{pubDate}: fecha en la que se publicó.
        \end{itemize}
        \item Relaciones
        \begin{itemize}
            \item \underline{user}: relación \underline{ForeignKey} con \underline{Profile}.
            \item \underline{item}: relación \underline{ForeignKey} con \underline{Item}.
        \end{itemize}
    \end{itemize}
    \item \underline{Keyword}: entidad que representa a una palabra clave y si esta caracteriza a un usuario, a una noticia o a ambos.
    \begin{itemize}
        \item Atributos
        \begin{itemize}
            \item \underline{term}: término.
        \end{itemize}
        \item Relaciones
        \begin{itemize}
            \item \underline{users}: relación \underline{ManyToMany} con \underline{Profile}.
            \item \underline{items}: relación \underline{ManyToMany} con \underline{Item}.
        \end{itemize}
    \end{itemize}
\end{itemize}