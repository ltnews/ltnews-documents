% !TEX root = ../../proyect.tex

\chapter{Estudio}\label{estudio}
\section{Metodología LEAN}\label{sec:metodologia_lean}

A medida que va acabando el siglo XX y empezando el XXI, los empresarios van siendo conscientes de que los métodos tradicionales para construir empresas han quedado anticuados. Ya no sirven aquellas reuniones interminables con acreedores para que financien proyectos, entre otros motivos, porque no hay sustento para estas. Además, la filosofía de institución de sociedades cambia de una versión más monolítica a una más ágil: las start-up.

Uno de los padres del nuevo método de emprendimiento es Steve Blank, y él es el que se cae en la cuenta y reformula la solución. Antes, para construir un producto, era necesario contar con una empresa, a la que cualquier banco estaba dispuesto a ceder un crédito. Sin embargo, ahora, se empiezan a construir empresas para desarrollar productos, por lo que no se posee ningún aval a priori.

Va pasando el tiempo, Steve Blank va enseñando su manera de emprender y aparecen dos personajes que van poblando el mismo bosque aunque con diferentes árboles. Estos son Eric Ries y Alex Osterwalder. El primero aplica la metodología LEAN al emprendimiento. El segundo, convierte el plan de negocio de una empresa a un gráfico sencillo de implementar y mostrar. A la par de estos acontecimiento, Blank llega hasta un concepto revolucionario dentro del emprendimiento: el desarrollo de los clientes.

Pues es en todo este marco de trabajo donde nace la teoría-práctica de Ash Maurya y su \textit{Running Lean}. Esta metodología no hace más que unir los tres conceptos anteriormente mencionados: \textit{Customer Development}, \textit{Bussiness Model Canvas} y \textit{Lean Startup}. Sin embargo, consigue algo que ninguno de los tres si quiera empezó: llevarlo a la práctica de una manera clara, concreta y sencilla.

Ash hace en su libro un rápido plan de acción para llevar a cabo el método LEAN a la construcción de una empresa que triunfe siguiendo una serie de procesos. Estos, se pueden resumir en el gráfico \ref{fig:running_lean} que aparece en su libro. Así como la figura muestra un resumen del proceso a seguir, el resumen del libro se puede encontrar en la siguiente frase del mismo, aparecida en el primer capítulo del libro.

\begin{quote}
	\small Running Lean is a systematic process for iterating from Plan A to a plan that works, before running out of resources.
	\begin{flushright}
		\citeA{running_lean}
	\end{flushright}
\end{quote}

En definitiva, no es más que documentar el plan inicial que se posee de la idea, identificar los posibles riesgos e ir sistemáticamente evolucionando nuestro plan hasta dar con el que el cliente comprará. Todo esto, como dice la cita, sin llegar a agotar los recursos. ¿Para qué es necesaria todas estas fases? Para conocer cuáles son los intereses reales del cliente. Parece trivial, pero no es sencillo: lleva tanto tiempo porque el mismo cliente no sabe qué problema posee.

\subsection{\textit{Running Lean}}

Una vez visto un enfoque general de la metodología LEAN, se verá cómo funciona operativamente el \textit{Running Lean} de Ash Maurya. Él distingue tres etapas dentro de la metodología, por lo que se comentará brevemente cada una de las fases.

\figura{}{img/lean/running_lean}{Resumen Running Lean}{fig:running_lean}{}

\subsubsection{Documentar el plan A}

La primera es la documentación de un plan inicial. Esto no es más que plasmar en un papel las ideas que se tienen sobre el producto. En vez de realizar un documento típico con todas sus partes y rellenando cada una de las casillas que definen un producto tipo, Ash proporciona un \textit{Lean Canvas}. Este es la evolución del \textit{Bussiness Canvas} de Osterwalder. En este se rellenan los atributos imprescindibles que definirán el producto con una condición: ha de ser un prototipo fácil de realizar y rápido en el tiempo.

Así, en un simple papel se puede realizar este y empezar a trabajar sobre el mismo. ¿No sirve? A tirarlo y a empezar de nuevo. Además que en este es posible ver de un simple vistazo las partes claves de la idea de negocio y por ello compartirlo para pedir opinión. Esto es fundamental, ya que la opinión del emprendedor con es imparcial. Este será una de las personas que más sepa del tema, y no será totalmente objetivo.

Este se rellena siguiendo un orden establecido, que ayudará a formar la cabeza de emprendedor para centrarse realmente en lo importante. La plantilla es la que se muestra a continuación.

\figura{}{img/lean/lean_canvas}{Lean Canvas}{fig:lean_canvas}{}

Las partes más importantes del \textit{canvas} son los tres primeros puntos: problema, segmento de clientes y propuesta de valor única. Tan es así, que Ash afirma que:

\begin{quote}
	\small Investors, and more important, customers, identify with their problems and don't care about your solution (yet). Entrepreneurs, on the other hand, are naturally wired to look for solutions. But chasing after solutions to problems no one cares enough about is a form of waste.
	\begin{flushright}
		\citeA{running_lean}
	\end{flushright}
\end{quote}

Es por tanto primordial encontrar el problema a solucionar, algo no trivial, porque debemos afinar en la necesidad real del cliente para así satisfacerla.

\subsubsection{Identificar los riesgos del plan}

Una vez se posee el primer esbozo de la idea, con una primera aproximación a las partes esenciales del producto y cómo será la venta de este en el mercado, se han de identificar unos riesgos iniciales. Todavía no se ha llegado a la madurez de la idea, pero es conveniente conocer ya los puntos flacos de esta.

Ash Maurya, de nuevo, ayuda con esto. Lo hace enseñando cuáles son las etapas por las que va a pasar el producto. Además, en cada una de estas, para que no se pierda el foco, muestra qué elementos se han de priorizar. Esta se puede ver en el siguiente gráfico.

\figura{}{img/lean/lean_market}{Adecuación de nuestra idea al mercado}{fig:lean_market}{}

Para llegar a esos riesgos iniciales sin saber si quiera cuándo ni cómo se van a afrontar, Ash ilustra a los futuros emprendedores con las siguientes preguntas.

\begin{itemize}
    \item \textbf{Problem/Solution Fit}: \textit{Do I have a problem worth solving?}
    \begin{itemize}
        \item \textit{Is it something customers want? (must-have)}
        \item \textit{Will they pay for it? If not, who will? (viable)}
        \item \textit{Can it be solved? (feasible)}
    \end{itemize}
    \item \textbf{Problem/Market Fit}: \textit{Have I built something people want?}
    \item \textbf{Scale}: \textit{How do I accelerate growth?}
\end{itemize}

Se centra sobre todo en el primer punto porque, como recomienda, hay que pivotar para llegar al problema concreto a solucionar. Esto ayudará cuando se pruebe el producto en la siguiente fase. Pero ya, contestando a las preguntas, es posible ver los riesgos a los que el emprendedor se verá sometido. Estos serán mayoritariamente de cinco tipos.

El primero hará referencia al problema. Este nos dirá cuánto \textit{dolor} le quita al posible comprador. Esto es importante, porque podemos ver un problema real, que ocurra a un número alto de personas. Sin embargo, si este es visto como algo accesorio, nunca se conseguirá venderlo. Hay que ver, por tanto, si el problema merece la pena afrontarlo por el nivel de importancia que le de el usuario.

El segundo consiste en conocer el modo de llegar al público objetivo y a través de qué canales. Al igual que antes, si hay un buen producto, que soluciona un problema concurrente, pero que el público objetivo percibe como algo que le hace \textit{sufrir}, no conseguirá llegar a ellos y, por tanto, no venderá.

El tercero tipo se relaciona con la estructura prevista de pérdidas y ganancias. Una vez que madurada la idea y encontrado el público objetivo y los canales para llegar a él, se hace necesario comprobar cómo se sostendrá económicamente la idea. Por ello, es necesario analizar el margen de beneficio y la estructura de costes e ingresos.

El cuarto consistirá en el tamaño del mercado. Hay que comprobar cuál es el nicho de mercado y con quién hay de competir. Con esto, aseguraremos la audiencia y el objetivo a nivel de cuota de mercado.

El quinto y último a analizar será comprobar si técnicamente se puede implementar la solución propuesta. Solo con este criterio, se puede cambiar o alterar ligeramente la idea de negocio.

\subsubsection{Probar sistemáticamente el plan}

Ya tenemos el \textit{canvas} y los posibles riesgos asociados. Ahora falta ir perfilándolo para llegar la madurez completa del producto. Las fases por las que pasará este testeo sistemático son las siguientes.

\figura{0.5}{img/lean/lean_learn}{Prueba sistemática}{fig:lean_learn}{}

Esta es la genialidad de Ash Maurya: el cómo convertir una metodología etérea en un algoritmo a seguir. Aquí unifica las ideas de Blank y Ries, porque es llevar la oficina a la calle. Esto no es otra cosa que preguntar y asentar ideas, preguntar y cambiar conceptos, preguntar y llegar a lo que la gente quiere.

Es por tanto primordial en esta fase salir a la calle y hacer entrevistas, muchas entrevistas. Estas llevarán al emprendedor a cambiar continuamente su  \textit{canvas} para acercarse más a la idea final.

Esta última parte contiene a su vez cuatro fases, que el emprendedor ha de ir quemando para llegar a la siguiente con algo cada vez más claro. En cada una de estas se ha de tener la mentalidad de aprender siempre algo.

Primero serán las entrevistas de problema. En estas se han de validar las hipótesis sobre el par: problema y segmento de cliente. De manera gráfica, Maurya propone \textit{aprender} en esta fase lo siguiente.

\cuadro{*2c}{Aprendizaje entrevista de problema}{tab:problem_interview}
{
    
    \textbf{Risk} & \textbf{Learn} \\
    Product risk: What are you solving? & How do customers rank the top three problems? \\
    Market risk: Who is the competition? & How do customers solve these problems today? \\
    Customer risk: Who has the pain? & Is this a viable customer segment? \\
}

Una vez aprendido esto, se podrá saltar al siguiente punto. Esto se cristaliza en el hecho de poder identificar al \textit{early adopter}, es decir, el primer usuario; si se tiene un problema que merezca la pena analizar; y si se puede averiguar cómo los usuarios potenciales resuelven a día de hoy este problema.

Una vez conseguido esto, se habrá llegado a las entrevistas de solución. Poseen la misma dinámica que la anterior, solo que ahora cambia lo que es necesario aprender. Además, hay de tener en cuenta que se está en la antesala de la salida a ventas del producto.

\cuadro{*2c}{Aprendizaje entrevista de solución}{tab:problem_solution}
{
    
    \textbf{Risk} & \textbf{Learn} \\
    Customer risk: Who has the pain? & How do you identify early adopters? \\
    Product risk: How will you solve it? & What is the minimum feature set needed to launch? \\
    Market risk: What is the pricing model? & Will customers pay for a solution? \\
}

Llegados a este punto, se tendrán unas características mínimas de el producto, un precio al que el cliente está dispuesto a pagarte y un negocio alrededor del producto.

Ya se poseen constancia de problema y solución, por lo que es necesario implementar la solución. Ahora, con lo que el emprendedor ha aprendido, no interesa enrolarse en hacer una aplicación 100\% funcional: hay que implementar solo aquello imprescindible para que sea vendida. Con esto, se hace un aseguramiento en tiempo y dinero. Hay que llegar a este Mínimo Producto Viable (MVP en adelante) para comprobar si el cliente si la solución implementada es acorde al problema real del cliente. Una vez hecho, se hace necesario volver a las entrevistas, pero sobre este MVP. Ahora, el emprendedor ha de aprender lo siguiente.

\cuadro{*2c}{Aprendizaje entrevista del MVP}{tab:problem_mvp}
{
    
    \textbf{Risk} & \textbf{Learn} \\
    Product risk: What is compelling about the product? & Landing Page, Activation Flow, UVP  \\
    Customer risk: Do you have enough customers? & Channels \\
    Market risk: Is the price right? & Price \\
}

Llegados a este punto, se poseerá una página explicativa del producto (\textit{Landing Page}), el canal desde que el usuario llega a dicha página hasta que se interesa por el producto (\textit{Activation Flow}) y un MVP que cumpla el UVP definido. Si en las entrevistas se aprecia que ya convence, solo queda lanzarlo al exterior y ver si se tienen los canales y precio adecuados. Esta será la última fase, continua a lo largo del tiempo: medir la adecuación entre el producto y el mercado.

Aquí se aprecia si convence el producto y cuál es el modo de llegar a que los usuarios paguen. Ya los \textit{early adopter} habrán entrado en el flujo de clientes, por lo que se tienen que tirar de ellos para conseguir más usuarios. Además, ahora se recibirán continuas propuestas para mejorar el MVP que se harán necesarias analizarla su aplicabilidad.

\subsection{Aplicación real}

Como se puede observar, esta metodología es totalmente practicable a un proyecto real. De una parte, porque parte de la experiencia de varios autores ilustrados sobre el tema, llevando este a la práctica. De otra, porque posee una serie de pautas para aplicarlo en un entorno totalmente compartido.

Así, lo que se ha realizado en este proyecto, es la aplicación de la metodología LEAN, con su variante en el \textit{Running LEAN} para proyectos de emprendimiento, a una idea propia. Esta, primero, era una idea puramente técnica, y con la ayuda y guías de este marco de trabajo, ha terminado por convertirse en una idea de negocio, al que poder dedicarle jornadas y jornadas de trabajo para llegar de la rentabilidad básica del mismo a un beneficio alto.

La correcta aplicación y su interiorización para poder llevarla a cabo ha ido de la mano de dos hechos. El primero es la realización de las asignaturas \textit{Fundamentos de Innovación y Emprendimiento}, FEI, y \textit{Taller de Innovación y Emprendimiento}, TEI. A través de estas asignaturas del máster, hemos estudiado y llevado a la práctica todos estos conceptos.

En FEI y TEI, en concreto, teníamos que llevar un proyecto a consecución siguiendo las claves que propone Maurya en su libro \citeA{running_lean}. Yo, para poder entenderlo con todas sus variantes, he utilizado este proyecto, mi TFM, como idea de negocio. Así, he ido viendo problemas buscando su resolución, he recibido feedback de parte de los profesores y alumnos, he aprendido de los proyectos ajenos. Como se ve, podría ser la fiesta del saber en cuanto a emprendimiento se refiere.

El segundo hecho importante para llevar este Estudio a término ha sido la lectura y estudio del libro tantas veces citado a lo largo de este apartado. \textit{Running Lean} me ha enseñado que lo aprendido en las asignatura de Emprendimiento del Máster tienen bastante de fundamento real e innovador. Además, estudiado el libro he comprobado cómo llevar las teorías de Maurya hasta límites insospechados.

Así, a lo largo de los siguientes puntos, iré desgranando los pasos que se indican en \textit{Running Lean} y cómo los he llevado a la práctica en mi proyecto. Además, se indicarán una serie de conclusiones sacadas de la realización del mismo.

Como se puede intuir viendo lo anterior, no tendrá sentido aplicar totalmente la metodología ya que la consecución del TFM no es otra que el aprendizaje de tecnologías y conocimientos, y no tanto la venta de productos. Esto, aunque no descartado en un marco futuro, no se realizará en las iteraciones LEAN. Por tanto, se quedará el Estudio hasta el conocimiento certero de la problema y su posterior solución.


\section{Primer estudio}\label{sec:primer_estudio}

LT-News, como se ha dicho, es un lector de noticias inteligentes, capaz de detectar los gustos de los usuarios en base a su utilización de la aplicación, de relacionar noticias entre sí y de detectar las noticias más importantes del día.

El objetivo principal del proyecto es filtrar las noticias de los periódicos y quedarse solo con aquellas realmente interesantes para el usuario. El objetivo secundario y subyacente al anterior no es otro que el de no perder el tiempo.

\figura{}{img/canvas/canvas_inicial}{LEAN Canvas inicial}{fig:canvas_inicial}{}

\subsection{Segmento de clientes}
El cliente objetivo es lector actual de noticias, joven o de mediana edad con cultura e interés sobre temas de actualidad y con manejo de las nuevas tecnologías. 

\subsection{Problema}
El top tres de problemas son los siguientes. Primero la gran cantidad de medios a elegir, por tanto, de noticias a leer. La segunda es que las noticias a leer noticias no son objetivas, es decir, hay las noticias falsas y el sesgo de los periódicos. La tercera es que, entre tanta cantidad de noticias, se pierden las noticias que realmente interesan.

\subsection{Propuesta de valor único}
La UPV es el algoritmo de inteligencia artificial capaz de extraer el perfil del usuario y de las noticias y de relacionarlas entre sí. Dicho de otro modo, ahorrar tiempo a la vez que estar al día.

\subsection{Principales hipótesis}
\begin{itemize}
    \item A los lectores de noticias les preocupa la gran cantidad de noticias.
    \begin{itemize}
        \item Es la hipótesis más arriesgada porque si la respuesta es que no, hay que pivotar completamente.
    \end{itemize}
    \item A los lectores de noticias les preocupa que las noticias no sean objetivos.
    \begin{itemize}
        \item Es una hipótesis que requiere ser probada para seguir validando el producto, aunque es una funcionalidad secundaria.
    \end{itemize}
    \item A los lectores de noticias les interesa estar al día de sus temas favoritos.
    \begin{itemize}
        \item Es otra hipótesis secundaria para comprobar.
    \end{itemize}
\end{itemize}

\section{Iteraciones}\label{sec:iteraciones}

El resumen de las iteraciones realizadas sobre el producto se encuentra en esta tabla.

\cuadro{*5c}{Iteraciones realizadas}{tab:prueba}
{
    
    \textbf{Iteración} & \textbf{Hipótesis} & \textbf{Fechas} & \textbf{Tests} & \textbf{Resultado} \\
    Problema 1 & Primera hipótesis & S/29-09 & 2 entrevistas & No aplica \\ 
    Problema 2 & Primera hipótesis & S/06-10 & 4 entrevistas & Se confirma \\ 
    Problema 3 & Segunda hipótesis & S/13-10 & 4 entrevistas & Se confirma \\ 
    Problema 4 & Tercera hipótesis & S/20-10 & 143 encuestas & No se confirma \\ 
    Solución 1 & Soluciones & S/27-10 & 4 entrevistas & Se confirma \\ 
}

El principal pivote realizado es simplificar el Producto Mínimo Viable, ya que hay un problema que no afecta de igual manera a los early adopters, por tanto, aunque sea importante, no se incluirá en la primera versión. Todos los demás problemas se han confirmado al cien por cien.

Con respecto a las soluciones, se han confirmado, aunque los usuarios piden más características. Dado que en la primera versión ya se ha demostrado un MVP aceptable por los usuarios, se tomarán estas características para futuras versiones.

En relación a los \textit{early adopters}, he cambiado su percepción. Antes los consideraba como personas cultas interesadas por lo que ocurre en la actualidad y que siguen una serie de temas. Ahora, veo que es cualquier que ya use un lector RSS o aplicación de un periódico. En concreto, me centro en los que usen Feedly.

\subsection{Primera iteración}

La primera iteración consistió en confirmar la primera hipótesis, es decir, confirmar el primer problema. Para ello, se hicieron dos entrevistas. De ellas no se sacaron ninguna conclusión, simplemente se redefinió lo que se consideraba por \textit{early adopters}. Hasta este momento eran jóvenes, por interés por la cultura y con gran uso de smartphone y ordenadores. Fue entonces cuando comprobé que estos eran más bien personas que ya usan un lector RSS o, sin más, una aplicación de información en su móvil.

Son estos mis \textit{early adopters} por un motivo: a estos usuarios les interesa estar al día de diferentes medios y ya han buscado una solución para estar al día, que no es otra que utilizar un lector RSS. Además, si llevan tiempo utilizándolo, se habrán dado cuenta del principal problema de este: la enorme cantidad de noticias a leer.

\figura{}{img/canvas/canvas_it_1}{LEAN Canvas primera iteración}{fig:canvas_it_1}{}

\subsection{Segunda iteración}

Viendo lo anterior, se ha decidió volver a preguntar lo mismo, pero a los verdaderos usuarios potenciales. Como se dijo, fue importante confirmar la primera hipótesis, ya que de esta partía todo el problema, y, por ende, todo el proyecto. Es por ello que se realizaron cuatro entrevistas siguiendo el siguiente esquema:

\begin{enumerate}
    \item Welcome
    \begin{enumerate}
        \item Presentación personal
        \item Qué quiero de esta entrevista
    \end{enumerate}
    \item Collect Demographics
    \begin{enumerate}
        \item Nombre
        \item Email
        \item Edad
        \item Código Postal
        \item Uso de apps de noticias
        \item Veces a la semana que lee el periódico
    \end{enumerate}
    \item Tell a Story
    \begin{enumerate}
        \item Contexto de la cantidad de noticias en la actualidad.
        \item Leer en papel vs Leer Digital: actualidad vs concentración.
    \end{enumerate}
    \item Problem Ranking
    \begin{enumerate}
        \item ¿Qué problemas tienes a la hora de estar al día de todas las noticias que te interesan? Dime en orden.
        \item ¿Piensas que las noticias que lees están dadas desde un punto de vista o pueden ser falsas? ¿Te preocupan?
        \item ¿Alguna vez haz visto una noticia de un tema que te interesaba y te ha dado pena no haberte enterado antes?
        \item ¿Usas alguna aplicación de noticias?
    \end{enumerate}
    \item Explore Customer Overview
    \begin{enumerate}
        \item Ver su opinión sobre la manera de leer noticias
    \end{enumerate}
    \item Wrapping Up
    \begin{enumerate}
        \item Te gustaría una aplicación donde: ¿se recojan todas las noticias, te las relaciona y te las recomienda?
    \end{enumerate}
    \item Document Results
    \begin{enumerate}
        \item Pasarlo a un Google Forms
    \end{enumerate}
\end{enumerate}


Sobre esto se ha conseguido confirmar la primera hipótesis de manera rotunda, aunque la segunda y la tercera no ha quedado del todo claras. Es claro que problema principal existe y la gente que ya lee en sindicadores de contenido ha intentado plantear soluciones en su día a día para solventarlo.

Sobre las entrevistas y el feedback recibido en clase, sí que se han modificado los canales. Antes eran un poco genéricos; ahora, intentan llegar más a los early adopter. Por tanto, se va a intentar que publiquen la aplicación en blog de aplicaciones, se van a mencionar en blogs y foros especializados en el tema y se va a intentar hacer SEO de la aplicación con los tags: aplicación, noticias, inteligente. Algunas de estas podrían ser:

\begin{itemize}
    \item El Androide Libre - \textsl{https://elandroidelibre.elespanol.com}
    \item Xataka - \textsl{https://www.xatakandroid.com/}
    \item Android4All - \textsl{https://andro4all.com/}
\end{itemize}

\figura{}{img/canvas/canvas_it_2}{LEAN Canvas segunda iteración}{fig:canvas_it_2}{}

\subsection{Tercera iteración}

La tercera iteración consistió en confirmar, de manera sistemática y similar a la anterior, las dos hipótesis restantes. Para ello, se volvieron a hacer cuatro entrevistas. De estas salió validada la segunda hipótesis y algo clara la tercera, es decir, la mitad lo afirmaron como problema y la otra mitad, no les parecía relevante.

Aquí también salieron algunas características adicionales para añadir al producto que se considerarán en el futuro, ya que no están dentro del producto mínimo viable y son las siguientes:

\begin{itemize}
    \item Sacar de una noticia su relación al tema de fondo o contexto. Para esto es necesario llegar a contenido de calidad.
    \item Una aproximación podría ser relacionar temas de las noticias con Wikipedia o fuentes relevantes y más profundas.
    \item Para móviles Android, hacer que la aplicación tenga Widgets, ya que algunos usuarios estaban interesados.
    \item Sobre los resultados de las entrevistas, hemos cambiado la formulación de los dos primeros problemas.
\end{itemize}

\figura{}{img/canvas/canvas_it_3}{LEAN Canvas tercera iteración}{fig:canvas_it_3}{}

\subsection{Cuarta iteración}

Como se ha dicho, después de haber validado las dos primeras hipótesis como Producto Mínimo Viable, quedaba la tercera. Dado que las entrevistas individuales en este caso resultaban ineficientes porque no había ninguna opinión que esclareciera el asunto, se decidió realizar una encuesta y pasarla por diferentes canales. Gracias a este medio se consiguieron tres cosas, en este orden de importancia:

\begin{itemize}
    \item confirmar la opinión sobre la tercera hipótesis,
    \item saber su futuro comportamiento con respecto a la aplicación,
    \item qué le ven en falta.
\end{itemize}

Se les preguntó, sacado de lo anteriormente, datos demográficos y confirmación de las tres hipótesis. No fueron solamente preguntas exclusivas, ya que no me darían ningún dato. Se les incluyo un lugar donde introducir texto sobre cómo intentan resolver su problema. Así se puede ver cuánto le interesa realmente eso que ha respondido.

El resumen fue el siguiente (la encuesta está en https://goo.gl/forms/UJHY6o6lV3JxmM712).

\figura{}{img/encuesta/a_edad}{Encuesta primera pregunta}{fig:a_edad}{}
\figura{}{img/encuesta/b_lectura}{Encuesta segunda pregunta}{fig:b_lectura}{}
\figura{}{img/encuesta/c_aplicacion}{Encuesta tercera pregunta}{fig:c_aplicacion}{}
\figura{}{img/encuesta/d_actualizado}{Encuesta cuarta pregunta}{fig:d_actualizado}{}
\figura{}{img/encuesta/e_fake}{Encuesta quinta pregunta}{fig:e_fake}{}
\figura{}{img/encuesta/f_contexto}{Encuesta sexta pregunta}{fig:f_contexto}{}
\figura{}{img/encuesta/g_temas}{Encuesta séptima pregunta}{fig:g_temas}{}
\figura{}{img/encuesta/h_otros}{Encuesta octavo pregunta}{fig:h_otros}{}


Como se pudo ver, no quedó confirmado la tercera hipótesis, por lo que se quitará del Producto Mínimo Viable y se añadirá como primera característica futura, ya que interesa a la mitad de los usuarios. Por otra parte, se reafirmaron las dos primeras hipótesis. Además, salieron de ahí diferentes opiniones, así como futuras características a considerar en un futuro. Algunos son:

\begin{itemize}
    \item Hacer una aplicación que consuma poco espacio.
    \item Hacer cómoda la lectura del cuerpo de la noticia.
    \item Evitar los problemas de conexión a internet.
\end{itemize}

Con esto, nos queda el siguiente Canvas:

\figura{}{img/canvas/canvas_it_4}{LEAN Canvas cuarta iteración}{fig:canvas_it_4}{}

\subsection{Quinta iteración}
Por último, una vez confirmados los problemas, se hicieron dos entrevistas de solución a usuarios potenciales de la aplicación, dispuestos a probar todas las versiones de este. Estos son realmente \textit{early adopters}, dispuestos a pagar en cuanto salga al mercado si ven un producto fiable.

Para esto, se procederá a enseñar una demo. Esta se hará primero con la \textit{Landing Page}. En esta se ha intentado mostrar las futuras características del proyecto, así como recoger datos de usuarios. Esta se encuentra en varios idiomas, con idea de empezar a distribuirla por personas de habla inglesa o incluso francesa, ya que se pasó la encuesta a amigos que viven en estos países.
Esta es la página resultante:

\figura{}{img/landing/landing_1}{Landing Page 1}{fig:landing_1}{}
\figura{}{img/landing/landing_2}{Landing Page 2}{fig:landing_2}{}
\figura{}{img/landing/landing_3}{Landing Page 3}{fig:landing_3}{}

Por otra parte, se le dio tráfico, para saber la opinión de los posibles usuarios.

\figura{}{img/analytics/analytics_1}{Google Analytics 1}{fig:analytics_1}{}
\figura{}{img/analytics/analytics_2}{Google Analytics 2}{fig:analytics_2}{}
\figura{}{img/analytics/analytics_3}{Google Analytics 3}{fig:analytics_3}{}

Con esto, se hicieron las entrevistas de solución. La estructura de la entrevista fue la siguiente:
\begin{enumerate}
    \item Welcome
    \begin{enumerate}
        \item Presentación personal
        \item Qué quiero de esta entrevista
    \end{enumerate}
    \item Collect Demographics
    \begin{enumerate}
        \item Nombre
        \item Email
        \item Edad
        \item Código Postal
        \item Uso de apps de noticias
        \item Veces a la semana que lee el periódico
    \end{enumerate}
    \item Tell a Story
    \begin{enumerate}
        \item Contexto de la cantidad de noticias en la actualidad
        \item Leer en papel vs Leer Digital: actualidad vs concentración
    \end{enumerate}
    \item Demo
    \begin{enumerate}
        \item Le enseño la Landing Page
        \item Hablamos de cómo voy a solucionar estos problemas y le pregunta su opinión
    \end{enumerate}
    \item Test Pricing
    \begin{enumerate}
        \item ¿Qué te parece el plan de pagos?
    \end{enumerate}
    \item Wrapping Up
    \begin{enumerate}
        \item ¿Me podrías dar tu email para seguir contándote cómo sigue el proyecto?
        \item ¿Conoces a alguien que pueda estar interesado en el proyecto?
    \end{enumerate}
    \item Document Results
    \begin{enumerate}
        \item Pasarlo a un Google Form
    \end{enumerate}
\end{enumerate}

En sendas entrevistas, se volvieron a confirmar la solución, explicando lo que se iba a realizar y cómo se implementaría. Pareció correcto, estando los \textit{early adopters} expectantes a futuras versiones.

\subsection{Quinta iteración}
Finalmente, validado ya el canvas, tanto a nivel de solución como de problemas, tanto cualitativa como cuantitativamente, ha variado en algunos aspectos que vamos a considerar. Todo lo demás, en definitiva, se ha explicado al principio del apartado.

El mayor cambio ha consistido en la variación de concepto del usuario potencial, esto gracias a las entrevistas realizadas. Otro gran cambio ha sido la formulación de problemas del usuario, que se han adaptado a un lenguaje más cercano a este, así como el evitar generalidades. De esto, como se ha visto, ha sido reducido para conseguir un MVP más conciso.

De las soluciones, se ha mantenido como estaba inicialmente, ya que ha sido validado por los usuarios. Esto se debe a que previamente a estos experimentos, se buscó las mejores soluciones posibles.

\figura{}{img/canvas/canvas_final}{LEAN Canvas final}{fig:canvas_final}{}

\section{Conclusión estudio}\label{sec:conclusiones_estudio}

La mayor conclusión no ha sido otra que el centrarse durante todo el proceso creativo y sistemático en el posible usuario. Esto ha sido posible gracias a entrevistas personales, cercanía con los posibles usuarios, adaptación del lenguaje técnico, así como otras técnicas sacadas por Ash Maurya en Running Lean.

De otra manera, personalmente, hubiese adoptado por otra técnica. Esta hubiese sido construir el producto como entendía e intentar después venderlo. Ante opiniones de usuarios, hubiese desarrollado sin ton ni son, solo por el simple hecho de vender.

Así, en cambio, sin haber construido nada, sin invertir tiempo ni cabeza en la implementación, ya vislumbro que el lo que comprará un posible usuario. Esto hace el trabajo más eficiente, así como más humilde. Humilde por el hecho de no canonizar mis ideas preconcebidas, sino pasarlas por el crisol de la opinión de los demás.