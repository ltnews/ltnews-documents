% !TEX root = ../../proyect.tex

\chapter{Introducción}\label{introduccion}
\section{Justificación del proyecto}\label{sec:justificacion}

Cuando decidí hacer mi Trabajo Fin de Grado, tome uno de los temas que más me apasiona: el periodismo. La idea era hacer una aplicación que superase a todas las existentes -se entiende que solo desde el punto de vista conceptual. Finalmente lo conseguí, quedando un lector RSS inteligente.

La aplicación cumplía todas las características que un usuario pide a un lector de noticias. Estas son: añadir cualquier fuente que posea sindicación de contenido, gestionar los \textit{feeds} y leer las noticias de estos al momento, llevando un control de lectura de las mismas. Además, poseía sobre esto añadidos interesantes como guardar noticias para leerlas más tarde o agrupar los \textit{feeds} en secciones.

Además de lo anterior, poseía características adicionales que le daban el sobrenombre de: inteligente. Una de estas era la capacidad de relacionar noticias entre sí y mostrárselas al usuario mientras leía un noticia. Otra era la de extraer el perfil del usuario y recomendarle noticias en base a cómo utilizase la aplicación y a qué tipo de artículos solía leer. La tercera característica era la búsqueda de noticias en base a los criterios que se desease.

Como se puede ver, no se encuentra actualmente un lector de noticias con estas características. ¿Por qué? El motivo es el coste de la infraestructura que ha de poseer la aplicación. Si pensamos en cualquier medio importante, veremos que su fichero RSS se actualiza de media cada diez minutos, publicando así una media de una quince noticias por hora. Si seguimos aritméticamente los cálculos, veremos que por día, sólo en España, se publican del orden de millones de artículos.

Por ello, las tecnologías, infraestructura y algoritmos que utilicé quedaron obsoletos al poco tiempo, siendo necesario profesionalizar el proyecto si se quería una aplicación que el público final pudiese utilizar. Además, faltaba en esta una parte importante, que era la monetización de la misma.

Como se puede decidir, el Trabajo Fin de Máster consistirá en rehacer la aplicación utilizando tecnologías adaptadas a esta enorme ingesta de datos y utilizando un formato más profesional, del que más tarde hablaremos. Además, de enfocar el proyecto hacia una aplicación comercial con la que sea posible ganar dinero.


\section{Objetivos}\label{sec:objetivos}

Los objetivos se engloban en dos categorías: funcionales y técnicos. En la primera se agrupan todos los conocimientos que quiero adquirir en cuanto a metodología, conocimiento del dominio del problema y venta de producto. Segundo será el aprender un determinado número de tecnologías o infraestructuras.


\subsection{Objetivos funcionales}

El objetivo principal a adquirir en cuanto a conocimiento es la experiencia emprendedora. Esto no es más que convertir de una idea tecnológica atrayente y estimulante, a una pequeña empresa que sostenga y monetice dicha idea. Aunque pueda parecer algo nimio a simple vista, requiere todo un proceso de concepción de la idea, el cliente y el ecosistema de ventas. Gracias a las asignaturas del máster, veo posible implementar dicha idea hacia una idea, ya no puramente tecnológica, sino de negocio.

Los objetivos subyacentes a esta son el conocimiento de las metodologías que me permitan realizar este proceso. En concreto, me he basado en tres ideas actuales de emprendimiento que se pueden encontrar a lo largo del documento y que se reducen a tres autores:

\begin{itemize}
	\item \textit{Customer Development} de Steve Blank
	\item \textit{Business Model} de Alex Osterwalder
	\item \textit{Running Lean} de Ash Maurya
\end{itemize}

En estos se definen los procesos de creación de la \textit{start-up}, creación del producto, atracción de clientes y de mejora continua. Me han parecido enormemente interesante y acertados para la aplicación en mi idea de negocio, por eso decidí aplicarlo.

El segundo objetivo es el conocimiento más férreo del dominio del problema de la idea de negocio, es decir, el periodismo. Está viviendo un proceso de cambio en cuanto al mundo de las noticias. Ya no interesa conocer lo que ocurrió el día anterior, sino lo que pasa ahora mismo en cualquier parte del mundo. Es por ello, que la prensa ha tenido que pivotar para saciar el ansia de información de sus usuarios. Además de esto, han tenido que atraer a usuarios más jóvenes, que ven más atrayente conocer el mundo a través de las redes sociales, antes que a través de periodistas.

En este proceso, además, se esconde una oscura pero cada vez más visible realidad: el mundo de la comunicación avanza varios pasos por detrás del ritmo \textit{impuesto} por la sociedad digital, por ello podemos ver múltiple deficiencias, que se explicarán a lo largo del documento. Es por ello también objetivo del presente estudio proponer mejoras al mismo.

\subsection{Objetivos técnicos}

El objetivo técnico por antonomasia a adquirir en el Trabajo Fin de Máster es desarrollar una aplicación siguiendo las pautas marcadas por las grandes compañías, en cuanto a tecnologías, herramientas e infraestructuras se refiere. Teniendo este preámbulo se entenderán correctamente estos objetivos.

El primer objetivo es el desarrollo de una aplicación utilizando la estrategia de desacople de capa de servidor y cliente. Esto no es más que ofrecer desde el \textit{backend} una API consumible desde cualquier cliente. Gracias a esto, se puede diferencias las tecnologías de cliente y servidor, haciéndolas independientes.

El segundo objetivo es desarrollar la capa de presentación utilizando un framework JavaScript. Esta estrategia es enormemente usada en el mundo del desarrollo web, ya que es posible realizar toda una aplicación \textit{frontend} con AJAX, es decir, sin recargas para extraer la información. Toda la comunicación con el servidor será asíncrona, por lo que la fluidez que percibirá el usuario será total.

El tercero será el desarrollo de una capa de inteligencia artificial que permita conocer tanto al usuario que interacciona con la aplicación así como el contenido que se muestra. Con ello, se podrá recomendar y extraer datos de ambos. Este, además de dar un valor añadido al desarrollo que se realice, lo hará enormemente interesante para un usuario con conocimiento del dominio del problema, ya que será posible extraer conclusiones.

Por último, y en la medida de lo posible, se incoará un proceso de integración y entrega continua. Para ello, se utilizarán herramientas que nos permitan agilizar en la medida de lo posible el tiempo desde que un desarrollador hace un cambio hasta que el usuario lo percibe en la web.


\section{Alcance}\label{sec:alcance}

Es objeto del presente trabajo es el estudio, análisis e implementación de un proceso completo de negocio que va desde la concepción de la idea hasta la venta de la misma. Para ello, se utilizará el método indicado por Ash Maurya con su idea del \textit{Running Lean}, enormemente apoyada por emprendedores de todo el globo.

Dentro de la implementación, realizado siempre un Producto Mínimo Viable, como indica Ash, se pretende realizar una lector RSS con características adicionales que le den un valor añadido como son el análisis de perfiles de usuarios, el análisis de medios y el análisis de noticias. El hacer hincapié en el \textit{análisis}, no es más que recordar lo que diferenciará nuestro producto del resto. El usuario, en definitiva, lo notará debido a la capacidad de realizar búsquedas en la aplicación, así como la recomendación de noticias.

Dentro del marco actual, la aplicación estará pensada para los dispositivos actuales en toda la gama de resoluciones de pantallas. Además, al permitirse la posibilidad de añadir cualquier tipo de \textit{feed}, se añadirá a la plataforma múltiples idiomas, en la medida de lo posible.


\section{Estado del arte}\label{sec:arte}

La sindicación de contenido vino a resolver un problema en 1995. Este no era otro que dar la posibilidad a un usuario de poder seguir desde una única plataforma todas las noticias de diferentes medios.

En cambio, si vemos como ha evolucionado el mundo digital, en cuanto al número de webs se refiere, podemos ver lo siguiente:

\begin{quote}
	\small En solo 10 años, hemos pasado de tener 92 millones de sitios web en Internet a más de 1.000 millones. Internet ya no es solo un lugar dónde encontrar información o dónde encontrar ocio y entretenimiento, ni es solo una herramienta de trabajo, Internet es dinero y es cambio social.
	\begin{flushright}
		\citeA{sitios_web}
	\end{flushright}
\end{quote}

Ahora, entonces, podemos ver otro problema, no resuelto todavía.