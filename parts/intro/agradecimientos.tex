\cdpchapter{Agradecimientos}

\hfill \textit{Omne agens agit sibi simile}.
\newline

Con este trabajo pongo punto y final a mi etapa académica. Finalmente, remarco los cuatro años de Grado y estos dos últimos de Máster. En total hacen seis años en la universidad, etapa que recordaré toda la vida. En ella he dejado recuerdos memorables y amigos indelebles... y, sobre todo, he madurado como persona diría que de manera integral. Es por ello, que en este apartado iré haciendo memoria de aquellas a las que debo esta etapa.

Estas reflexiones tienen más de epitafio, que de elegía a esta etapa que termina. Es por ello que dejaré que el corazón se explaye más que de transmitir aquello de lo que dejo constancia.

Creo que la antesala de la universidad es el Bachillerato. Tuve la suerte de cursar estos dos años (más uno anterior) en el Colegio Guadalete. En él aprendí de la mano de los mejores profesores y de muy buenos tutores. No solo debo agradecimiento a aquello que me enseñaron en las materias que luego me ayudarían, sino que intentaron darme una formación humanística anexa a la rama que escogiera: siempre la científica.

Aprendí mi base de Matemáticas gracias a Enrique Muñoz Estrada. Puedo decir que ha sido para mí el mejor profesor que he tenido. Me enseñó una forma de trabajo y entendimiento diferente a la que ya poseía, y con la que me quedé siempre. Al igual que con él, debo a José María Gallardo mis primeros conocimientos de Filosofía y mi primer acercamiento al arte. Decía Ludwig Wittgenstein que \textit{el lenguaje es el vehículo del pensamiento}. Creo que este es el motivo de que bebiera las explicaciones que nos daba acerca de aquellos fantásticos autores.

De igual modo a Bartolomé Hernández, Emilio Flores, José Barrera y Enrique Garrucho. Tanto los mencionados, como los que quedan en el tintero, me ayudaron a conseguir lo dicho en la charla de imposición de becas de aquel 16 de marzo de 2013 y que se ha quedado grabado en mi cabeza: \textit{cómo ser universitario y no morir en el intento}.

Por otro lado, el segundo lugar donde tuvo lugar mi formación durante este época fue en el Club Juvenil Mainel. Aquí me transmitieron aquella formación integral de que la que hablaba. Doy las gracias a Cristóbal Táuler, Ignacio Carriazo, José María Cruz y Juan Ignacio Antón. Todos estos marcaron un hito en vida, siempre a mejor.

Comienza septiembre de 2013 y me encuentro que estoy de lleno metido en la universidad, un ambiente de excelencia académica, mezclado con la inseguridad e irresponsabilidad propias de la edad. Durante los cuatro años de grado he tenido profesores muy buenos, de los que enseñan a ser Ingeniero e Informático. En concreto, quería citar a algunos.

En primer lugar y las primeras en orden de llegada, a Luisa María Camacho y Delia Garijo. Con ellas vi lo que era aquella matemática que siempre me ha fascinado. Nunca olvidaré aquel éxtasis interior al comprender aquello casi ininteligible que explicaban con tanta facilidad y dulzura. Aunque más tarde, ya en tercero, obró igual efecto en mí María del Rocío Moreno.

Dentro del campo propio de la ingeniería informática, fue Francisco Ferrer Troyano. Sin él, seguramente, la visión que tengo actualmente de la programación hubiese sido otra. Al igual, con Miguel Toro y sus clases de algoritmia y datos: \textit{pon un Map en tu vida}; a Alejandro Fernández-Montes y las prácticas con mi primer framework GWT, explicadas con maestría y cercanía; a Manuel Rovayo y sus explicaciones sobre la historia de la informática, de las que sobreabunda conocimiento y ansias de transmitirlo; a Vicente Carillo y sus tutorías en el TFM sobre un tema que tanto me apasionara.

Aunque se mencionan algunos nombres, quiero agradecer a todos aquellos que me transmitieron conocimiento a través de cada asignatura, cada práctica, cada seminario. Puedo decir que no he tenido profesores malos: simplemente algunos que explicaban peor, pero siempre intentando enseñar.

Creo que esta etapa en la universidad habría sido diferente si no hubiese tenido al lado a ese gran Colegio Mayor Almonte. Allí residía y vivía. Allí me enseñaron a estudiar y a ser cada vez mejor persona. Allí adquirí esa vena humanística que espero nunca perder, gracias a todas aquellas sesiones sobre literatura, cine, poesía, pensamiento... cultura en definitiva.

Sobre todo esos tres veranos que pasé con el Colegio Mayor en Barcelona serán para mí memorables al paso de los años. Esos cursos de filosofía me expandieron intelectualmente hacia aquellos otros campos, más allá de la ingeniería.

De esta etapa, como decía, tengo que agradecer a muchas personas. Gracias a José Luis León y Juan Carlos Domínguez; a Ignacio Díez y Nacho Bárcena; a Juan Soler y Juanjo de Paiz. De todos he ido aprendiendo poco a poco, \textit{ese no se qué} que quedará siempre dentro y que me hicieron madurar.

Después de estos dos años grandiosos, decidí dar todo aquello que había recibido. Por eso, participaba como parte activa de la formación integral de los jóvenes en el Club Juvenil Arqueros. Hasta ahora, he podido conocer a gente fantástica, de las que he aprendido enseñando. Además, junto al equipo de formadores, he disfrutado, madurado y aprendido. Gracias Jaime Arana, Josepmaria Quintana, Ángel Luis Gómez: habéis terminado de enseñarme, prácticamente, cosas que ya entreveía, así como terminar esta etapa \textit{Chapeau!}.

Beatriz Bernárdez y sus sesiones de Mindfullness y Bases de Datos: como ya te he dicho alguna vez, te 

