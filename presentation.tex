% Definición
\documentclass[12pt]{beamer}

% Paquetes
\usepackage{graphicx,listings}
\usepackage[utf8]{inputenc}
\usepackage[spanish,es-tabla]{babel}
\usepackage{times}          % Usar tipo Times-Roman
\usepackage[T1]{fontenc}    % Usar la codificación T1

% Datos
\title{Metodología Running Lean aplicada a un lector de noticias inteligente}
\author{Andrés M. Jiménez Ríos}
\institute[TFM]{Trabajo Fin de Máster}

% Temas
\usetheme{Boadilla}
\usecolortheme{crane}
\useoutertheme{infolines}
\useinnertheme{rectangles}

% Opciones
\setbeamertemplate{navigation symbols}{}

\lstset{
	literate={á}{{\'o}}1
	{é}{{\'e}}1
	{í}{{\'i}}1
	{ó}{{\'o}}1
	{ú}{{\'u}}1
}


\AtBeginSection{ 
	\begin{frame}{Índice} 
		\tableofcontents[currentsection]
	\end{frame}
}

\AtEndDocument{
	\frame{\titlepage}
}

% Inicio
\begin{document}

% Diapositivas
	\frame{\titlepage}
	
	\begin{frame}{Índice}
		\tableofcontents
	\end{frame}

	\section{Introducción}	
		\begin{frame}{\textit{Running Lean}}
			\begin{block}{}
				\textit{Running Lean is a systematic process for iterating from Plan A to a plan that works, before running out of resources.}
			\end{block}
			\begin{block}{Influencia}
				\begin{itemize}
                    \item Steve Blank - \textit{Customer Development}
                    \item Eric Ries - \textit{Lean Startup}
                    \item Alex Osterwalder - \textit{Bussiness Model Canvas}
                \end{itemize}
			\end{block}
        \end{frame}

        {
            \usebackgroundtemplate{\includegraphics[height=\paperheight,width=\paperwidth]{img/lean/running_lean}}
            \setbeamertemplate{navigation symbols}{}
            \begin{frame}[plain]
            \end{frame}
        }
	
		\begin{frame}{Periodismo}
			\begin{block}{Situación actual}
				La falta de contenido de calidad, las redes sociales y las \textit{fake news}.
			\end{block}
			\begin{block}{Mundo digital}
				Su propuestas son la sindicación de contenido, las redes sociales y aplicaciones propias.
			\end{block}
		\end{frame}

		\begin{frame}{Propuesta}
			\begin{block}{}
				Estudio y aplicación de la metodología \textit{Running Lean}.
			\end{block}
			\begin{block}{}
				Análisis de los problemas actuales del periodismo digital.
			\end{block}
			\begin{block}{}
				Realización de una aplicación que implemente las propuestas.
			\end{block}
		\end{frame}
	
	
	\section{Estudio}
		\begin{frame}{Canvas inicial}
            \includegraphics[width=\textwidth,height=0.8\textheight,keepaspectratio]{img/canvas/canvas_inicial}
        \end{frame}

        \begin{frame}{Hipótesis de problemas}
            \begin{itemize}
                \item A los lectores de noticias les preocupa la gran cantidad de noticias.
                \item A los lectores de noticias les preocupa que las noticias no sean objetivas.
                \item A los lectores de noticias les interesa estar al día de sus temas favoritos.
            \end{itemize}
        \end{frame}

        \begin{frame}{Iteraciones realizadas}
            \resizebox{\textwidth}{!}{
                \begin{tabular}{*5c}
                    \hline
                    \textbf{Iteración} & \textbf{Hipótesis} & \textbf{Fechas} & \textbf{Tests} & \textbf{Resultado} \\
                    Problema 1 & Primera hipótesis & S/29-09 & 2 entrevistas & No aplica \\ 
                    Problema 2 & Primera hipótesis & S/06-10 & 4 entrevistas & Se confirma \\ 
                    Problema 3 & Segunda hipótesis & S/13-10 & 4 entrevistas & Se confirma \\ 
                    Problema 4 & Tercera hipótesis & S/20-10 & 143 encuestas & No se confirma \\ 
                    Solución 1 & Soluciones & S/27-10 & 4 entrevistas & Se confirma \\ 
                    \hline
                \end{tabular}
            }
        \end{frame}
        
		\begin{frame}{Canvas final}
            \includegraphics[width=\textwidth,height=0.8\textheight,keepaspectratio]{img/canvas/canvas_final}
        \end{frame}
\end{document}